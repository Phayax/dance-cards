\documentclass[
	12pt,
	%draft, % show overfull hboxes
	]{scrartcl}

\usepackage[    % set up dimensions
	a6paper, 
	landscape,
	left=1.45cm,   
	right=1.45cm,  
	top=1.6cm,
	bottom=1.6cm, 
	%showframe,
	nomarginpar,  % this removes the right margin, centering the text in the middle
	noheadfoot,  % this removes the head and foot margin
	]{geometry}

%\pagestyle{empty}

\usepackage{graphicx}
\usepackage{tikz}
\usetikzlibrary{external}
\usepackage{eso-pic}

% Fonts and encodings
\usepackage[nf]{coelacanth}
\usepackage[T1]{fontenc}

\usepackage{longtable} % Um Tabellen zu machen, die mehrere Seiten lang sind und einen passenden Seitenumbruch haben.
\usepackage{multicol}
%\usepackage[singlespacing]{setspace} % Zeilenabstand
\usepackage{nicefrac} % für besser aussehende Brüche

\usepackage[object=vectorian]{pgfornament}

%\tikzexternalize[prefix=tikzext/]
%\tikzset{external/force remake}


\AddToShipoutPictureBG{%
\begin{tikzpicture}[remember picture, overlay]
	% pgfornament does give very simple and nice vectorian graphics for tikz.
	% The problem is that the lines are too thick and figure 41 is also slightly cropped.
	% pst-vectorian works very well.
	% we're now using the original files from vectorian exported as eps file
	
	% DEBUGGING grid
	%\draw[line width=0.75pt, color=teal, step=.5cm, xshift=0.08cm, yshift=0.43cm] (current page.south west) grid ++(14.85cm,10.5cm);
	% DEBUGGING corners
	%\node [circle, fill=red, minimum size=1pt, xshift=0.0cm, yshift=-0.0cm, scale=0.2] at (current page.north west) {};
	%\node [circle, fill=red, minimum size=1pt, xshift=0.5cm, yshift=-0.5cm, scale=0.2] at (current page.north west) {};
	
	
	\node[anchor=north west, xshift=0.5cm, yshift=-0.5cm, inner sep=0] at (current page.north west){\includegraphics[scale=0.8, angle=0, origin=c]{img/corner_black}};
	\node[anchor=north east, xshift=-0.5cm, yshift=-0.5cm, inner sep=0] at (current page.north east){\includegraphics[scale=0.8, angle=0, origin=c]{img/corner2_black}};
	\node[anchor=south west, xshift=0.5cm, yshift=0.5cm, inner sep=0] at (current page.south west){\includegraphics[scale=0.8, angle=180, origin=c]{img/corner2_black}};
	\node[anchor=south east, xshift=-0.5cm, yshift=0.5cm, inner sep=0] at (current page.south east){\includegraphics[scale=0.8, angle=180, origin=c]{img/corner_black}};
	
	
	% Original pgfornament style:
	%\node[anchor=north west, xshift=0.5cm, yshift=-0.5cm,inner sep=0] at (current page.north west){\pgfornament[scale=0.5]{41}};
	%\node[anchor=north east, xshift=-0.5cm, yshift=-0.5cm, inner sep=0] at (current page.north east){\pgfornament[scale=0.5,symmetry=v]{41}};
	%\node[anchor=south west, xshift=0.5cm, yshift=0.5cm, inner sep=0] at (current page.south west){\pgfornament[scale=0.5,symmetry=h]{41}};
	%\node[anchor=south east, xshift=-0.5cm, yshift=0.5cm, inner sep=0] at (current page.south east){\pgfornament[scale=0.5,symmetry=c]{41}};
	
	% Background as an image, for example a parchment like texture
	%\tikz[remember picture, overlay] \node[opacity=0.5, inner sep=0pt] at (current page.center){\includegraphics[width=\paperwidth,height=\paperheight]{example-image}};
	
\end{tikzpicture}
}

\newcommand{\danceinfo}[1]{%
	{\raggedleft\footnotesize{#1}\par}
}
\newcommand{\dancename}[1]{
	% Dimensionen ausprobiert, dass sie nicht mit den 41er Ornamenten in den Ecken kollidieren.
	
	\begin{tikzpicture}[remember picture, overlay]
%		\node[] at (0, 0) (title) {Text};		
		\node[anchor=north, yshift=-0.5cm, inner sep=0, text width=10.5cm, align=center] at (current page.north){\Huge{#1}};
	\end{tikzpicture}
}
\newcommand{\origininfo}[3]{
	% Dimensionen ausprobiert, dass sie nicht mit den 41er Ornamenten in den Ecken kollidieren.
	\begin{tikzpicture}[remember picture, overlay]
		% setting the font size in the node options, since it doesn't seem to 
		% take effect when used in the node text. 
		\node[anchor=south west, xshift=1.3cm, yshift=1.15cm, inner sep=0, align=left, font={\scriptsize}] at (current page.south west){Choreographie:\\{#1}};
		\node[anchor=south east, xshift=-1.2cm, yshift=1.15cm, inner sep=0, align=right, font={\scriptsize}] at (current page.south east){Musik:\\ {#2}\\ {#3}};
	\end{tikzpicture}

%	\begin{textblock}{4}(1.3,12.6)
%		\begin{flushleft}
%			{\scriptsize Choreographie:\\[-0.2cm] {#1}}
%		\end{flushleft}
%	\end{textblock}
%	
%	\begin{textblock}{4}(10.7,12.2)
%		\begin{flushright}
%			{\scriptsize Musik:\\ {#2}\\[-0.2cm] {#3}}
%		\end{flushright}
%	\end{textblock}
}
\newcommand{\danceeasymarker}{
	\begin{tikzpicture}[remember picture, overlay]
		\node[anchor=south, xshift=0cm, yshift=0.5cm, inner sep=0] at (current page.south){\pgfornament[width=4cm]{80}};
	\end{tikzpicture}
}
\newcommand{\dancemediummarker}{
	\begin{tikzpicture}[remember picture, overlay]
		\node[anchor=south, xshift=0cm, yshift=0.5cm, inner sep=0] at (current page.south){\pgfornament[width=5cm]{83}};
	\end{tikzpicture}
}
\newcommand{\dancedifficultmarker}{
\begin{tikzpicture}[remember picture, overlay]
	\node[anchor=south, xshift=0cm, yshift=0.4cm, inner sep=0] at (current page.south){\pgfornament[width=5cm]{84}};
\end{tikzpicture}
}
\newcommand{\danceinstructionsbegin}{\begin{longtable}{p{1cm}p{9.8cm}}}
\newcommand{\danceinstructionsend}{\end{longtable}}
\newcommand{\danceinstructionsel}{~ & ~ \\}

\setlength{\topskip}{12pt}
\setlength{\parskip}{0pt}

\pagestyle{empty}

\begin{document}

%%%%%%%%%%%%%%%%%%%%%%%%%%%%%%%%%%%%%%%%%%%%%%%%%%%%%%%%%%%%%%%%%%%%%%%%%%%%%%%%
% Allemande (Paartanz)
%%%%%%%%%%%%%%%%%%%%%%%%%%%%%%%%%%%%%%%%%%%%%%%%%%%%%%%%%%%%%%%%%%%%%%%%%%%%%%%%
\newpage

\dancename{Allemande}
%\subsection{Allemande}
\iffalse\subsection{Allemande (Paartanz)}\fi % fool texstudio into displaying subsections
\danceinfo{Paartanz}
\textit{Die Dame steht vor dem Herren in der Kiekbuschfassung}\\
\danceinstructionsbegin
1 -- 24 & 3x Chassé links und Chassé rechts, dann wenden\\
%\danceinstructionsel
1 -- 24 & 3x Chassé links und Chassé rechts zurück\\
& \textit{Linke Hände lösen, zueinander wenden}\\
\danceinstructionsel
1 -- 8 	& Das Paar tanzt seitwärts ein kleines Chassé links, dann ein Chassé rechts\\
1 -- 4 	& Schritt kick nach links und rechts\\
5 -- 8 	& Platzwechsel mit Handtour links\\
\danceinstructionsel
& \textit{Rechte Hände loslassen, linke Hände fassen}\\
1 -- 8 & Das Paar tanzt seitwärts ein kleines Chassé links, dann ein Chassé rechts\\
1 -- 4 	& Schritt kick nach rechts und links\\
5 -- 8 	& Die Dame dreht vor dem Herren und nimmt wieder Kiekbuschfassung an.
\dancemediummarker
\danceinstructionsend



%%%%%%%%%%%%%%%%%%%%%%%%%%%%%%%%%%%%%%%%%%%%%%%%%%%%%%%%%%%%%%%%%%%%%%%%%%%%%%%
%	Black Nag
%%%%%%%%%%%%%%%%%%%%%%%%%%%%%%%%%%%%%%%%%%%%%%%%%%%%%%%%%%%%%%%%%%%%%%%%%%%%%%%%
\newpage

\dancename{Black Nag}
\iffalse\subsection{Black Nag}\fi % fool texstudio into displaying subsections
\danceinfo{Longway for six}
%\dancemediummarker
%\begin{longtable}{p{1cm}p{10cm}}
\danceinstructionsbegin
\textbf{Strophe} & \textbf{~I}\\
1 -- 16 & Lead up and down (zweimal)\\
1 -- 12 & Paar 1, 2, dann 3 Seitgalopp nach oben\\
13 -- 16 & Drehung links\\
1 -- 12 & Paar 1, 2, dann 3 Seitgalopp nach unten\\
13 -- 16 & Drehung rechts\\
\danceinstructionsel
\textbf{Strophe} & \textbf{~II}\\
1 -- 16 & Siding links, Siding rechts\\
1 -- 4 & Platztausch H1 und D3 im Seitgalopp\\
5 -- 8 & Platztausch H3 und D1 ebenso\\
9 -- 12 & Platztausch Paar 2\\
13 -- 16 & Drehung rechts\\
1 -- 16 & Platztausche und Drehungen wiederholen\\
\danceinstructionsel
\textbf{Strophe} & \textbf{~III}\\
1 -- 16 & Armtour rechts, links\\
1 -- 12 & Hecke Herren \\
13 -- 16 & Drehung Damen links\\
1 -- 12 & Hecke Damen\\
13 -- 16 & Drehung Herren rechts\\
\dancemediummarker
\danceinstructionsend
{\footnotesize (Keine Wiederholungen)}


%%%%%%%%%%%%%%%%%%%%%%%%%%%%%%%%%%%%%%%%%%%%%%%%%%%%%%%%%%%%%%%%%%%%%%%%%%%%%%%%
% Blue Flag
%%%%%%%%%%%%%%%%%%%%%%%%%%%%%%%%%%%%%%%%%%%%%%%%%%%%%%%%%%%%%%%%%%%%%%%%%%%%%%%%
\newpage

\dancename{Blue Flag}
\iffalse\subsection{Blue Flag}\fi % fool texstudio into displaying subsections
\danceinfo{Kreistanz, Herr innen, Dame außen, ¾ Takt\\Partner anschauen, rechte Handfassung}
\danceinstructionsbegin
1--6 	& Balance vor zurück\\
7--12 	& Platzwechsel mit Drehung\\
1--12 	& Wiederholung\\
\danceinstructionsel
1--6 	& Balance vor,  zurück\\
7--9 	& Herr einen Schritt vor, Dame dreht  180° um links, Übergang in Kiekbuschfassung.\\
10--12	& Gemeinsam einen Schritt zurück\\
\danceinstructionsel
1--3 	& Gemeinsam einen Schritt nach vorne\\
4--6 	& Aus der Kiekbuschfassung lösen\\
7--9 	& Beide kicken gegen die Kreisrichtung\\
10--12 	& Dame dreht unter dem linken Arm des Herren einen Platz weiter\\
\danceeasymarker
\danceinstructionsend
\textit{Anmerkung: Auf drei Schläge kommt immer „Bewegung“, „Hoch“ und „Ab“.}


%%%%%%%%%%%%%%%%%%%%%%%%%%%%%%%%%%%%%%%%%%%%%%%%%%%%%%%%%%%%%%%%%%%%%%%%%%%%%%%%
% Branle Cassandra
%%%%%%%%%%%%%%%%%%%%%%%%%%%%%%%%%%%%%%%%%%%%%%%%%%%%%%%%%%%%%%%%%%%%%%%%%%%%%%%%
\newpage

%\dancemediummarker
\iffalse\subsection{Branle Cassandra}\fi % fool texstudio into displaying subsections
\dancename{Branle Cassandra}
\danceinfo{Kreistanz, Damen/Herren\\Paare zueinandergedreht}
\danceinstructionsbegin
1--4 	& zwei Anstellschritte links\\
5--8 	& Drehung nach rechts\\
9--12 	& zwei Anstellschritte rechts\\
13--16 	& Drehung links (Paare ungefähr rechte Schulter vor rechter Schulter)\\
\danceinstructionsel
1--4 	& Schulterbalancé rechts\\
5--8 	& Doppelhandfassung Platzwechsel (Herr innen)\\
9--10 	& Schritt links und klatschen\\
11--14 	& Drehung rechts\\
\danceinstructionsel
1--4 	& Schulterbalancé links\\
5--8 	& Doppelhandfassung Platzwechsel (Herr innen)\\
9--10 	& Schritt rechts und klatschen\\
11--14 	& Weitergehen zum nächsten andersgeschlechtlichen Partner\\
\dancemediummarker
\danceinstructionsend


%%%%%%%%%%%%%%%%%%%%%%%%%%%%%%%%%%%%%%%%%%%%%%%%%%%%%%%%%%%%%%%%%%%%%%%%%%%%%%%%
% Branle des Pois
%%%%%%%%%%%%%%%%%%%%%%%%%%%%%%%%%%%%%%%%%%%%%%%%%%%%%%%%%%%%%%%%%%%%%%%%%%%%%%%%
\newpage

\dancename{Branle des Pois}
\iffalse\subsection{Branle des Pois}\fi % fool texstudio into displaying subsections
\danceinfo{Kreistanz, Herr/Dame, durchgefasst}
\danceinstructionsbegin
1--8 	& Alle Double links, Double rechts\\
9--16 	& Wiederholen\\
\danceinstructionsel
1--2 	& Die Herren machen einen kleinen Hüpfer zur “fremden” Dame.\\
3--4 	& Die Damen hüpfen hinterher.\\
5--8	& Die Herren machen drei kleine Hüpfer zur “fremden” Dame.\\
1--2 	& Die Damen hüpfen hinterher.\\
3--4 	& Die Herren machen einen kleinen Hüpfer zur “fremden” Dame.\\
5--8 	& Die Damen hüpfen drei Mal hinterher.\\
\danceeasymarker
\danceinstructionsend


%%%%%%%%%%%%%%%%%%%%%%%%%%%%%%%%%%%%%%%%%%%%%%%%%%%%%%%%%%%%%%%%%%%%%%%%%%%%%%%%
% Branle des Rats
%%%%%%%%%%%%%%%%%%%%%%%%%%%%%%%%%%%%%%%%%%%%%%%%%%%%%%%%%%%%%%%%%%%%%%%%%%%%%%%%
\newpage

\dancename{Branle des Rats}
\iffalse\subsection{Branle des Rats}\fi % fool texstudio into displaying subsections
\danceinfo{Longway for as many as will (Kein Fortschritt)}
\danceinstructionsbegin
1--8 	& Alle Double links, Double rechts\\
9--16 	& Wiederholen\\
1--2 	& Die Tänzer springen mit dem linken Fuß schräg links aufeinander zu\\
3--4 	& Dann springen sie mit dem rechten Fuß schräg aufeinander zu\\
5--6 	& Mit einem weiteren Sprung mit links gehen sie Rücken an Rücken aneinander vorbei\\
7--8 	& Mit dem letzten rechten Schritt drehen sie sich fertig auf die gegenüberliegende Grundlinie\\
\danceeasymarker
\danceinstructionsend
\textit{Wiederholen und dabei die Geschwindigkeit der Musik anpassen}


%%%%%%%%%%%%%%%%%%%%%%%%%%%%%%%%%%%%%%%%%%%%%%%%%%%%%%%%%%%%%%%%%%%%%%%%%%%%%%%%
% Candles in the Dark
%%%%%%%%%%%%%%%%%%%%%%%%%%%%%%%%%%%%%%%%%%%%%%%%%%%%%%%%%%%%%%%%%%%%%%%%%%%%%%%%
\newpage

\dancename{Candles in the Dark}
\origininfo{Loretta Holz (2006)}{Candles in the Dark}{von Jonathan Jensen}
\iffalse\subsection{Candles in the Dark}\fi % fool texstudio into displaying subsections
\danceinfo{Longway for as many as will\\¾ Takt, Walzerschritte}
\danceinstructionsbegin
1--12 	& H1 D1 geführte Half-Figure Eight durch das andere Paar\\
1--12 	& H1 D2 geführte Half-Figure Eight durch das andere Paar\\
1--12 	& H2 D1 geführte Half-Figure Eight durch das andere Paar\\
1--12 	& H2 D2 geführte Half-Figure Eight durch das andere Paar\\
\danceinstructionsel
1--12 	& Mirror Dos-à-Dos auf der Linie - Paar 2 innen \\
1--12 	& ganzer Setkreis\\
\danceinstructionsel
1--12 	& Mirror Dos-à-Dos auf der Linie – Paar 1 innen\\
1--12 	& 1 ½ Ronden mit dem Partner\\
\dancemediummarker
\danceinstructionsend


%%%%%%%%%%%%%%%%%%%%%%%%%%%%%%%%%%%%%%%%%%%%%%%%%%%%%%%%%%%%%%%%%%%%%%%%%%%%%%%%
% Chapelloise
%%%%%%%%%%%%%%%%%%%%%%%%%%%%%%%%%%%%%%%%%%%%%%%%%%%%%%%%%%%%%%%%%%%%%%%%%%%%%%%%
\newpage

\dancename{Chapelloise}
\iffalse\subsection{Chapelloise}\fi % fool texstudio into displaying subsections
\danceinfo{Kreistanz, Herren innen, GUZS}
\danceinstructionsbegin
\textbf{I}&\\
1--4 	& Vier Schritte vor, auf 4 Drehung und Handwechsel\\
5--8 	& Vier Schritte rückwärts\\
1--8 	& Wieder zurück\\
\danceinstructionsel
\textbf{II}&\\
1--2 	& Zusammen hüpfen, „flirten“\\
3--4 	& Außeinander hüpfen, nach außen respektive innen schauen\\
5--8 	& Platzwechsel, Dame geht vorm Herren vorbei\\
1--2 	& Zusammen\\
3--4 	& Außeinander\\
5--8 	& Die Dame dreht unter dem linken Arm des Herren nach hinten (über ihre rechte Schulter)
\danceeasymarker
\danceinstructionsend



%%%%%%%%%%%%%%%%%%%%%%%%%%%%%%%%%%%%%%%%%%%%%%%%%%%%%%%%%%%%%%%%%%%%%%%%%%%%%%%%
% Circassian Circle
%%%%%%%%%%%%%%%%%%%%%%%%%%%%%%%%%%%%%%%%%%%%%%%%%%%%%%%%%%%%%%%%%%%%%%%%%%%%%%%%
\newpage

\dancename{Circassian Circle}
\iffalse\subsection{Circassian Circle}\fi % fool texstudio into displaying subsections
\danceinfo{Kreistanz, Herr/Dame, durchgefasst}
\danceinstructionsbegin
1 -- 8 	& Double vor und zurück\\
9 -- 16 & Wiederholung\\
\danceinstructionsel
1 -- 4 	& Damen Double vor, Klatschen auf 4\\
5 -- 8 	& Damen Double zurück, Herren Klatschen auf 5\\
1 -- 4 	& Herren Double vor\\
5 -- 8 	& Herren drehen sich nach links und gehen zu der Dame links von ihnen\\
\danceinstructionsel
1 -- 16 & Herr und Dame drehen: Rechten Füße aneinander setzen. Die rechten Hände werden auf die Schulter des Partners gelegt, die linken darunter gefasst. Dann mit den linken Füßen „abstoßen“ und so oft drehen wie beliebt.\\
1 -- 12 & Die Dame dreht sich vor dem Herren, sodass sie in der Kiekbuschfassung stehen und schreiten 12 Schläge auf der Kreisbahn entlang\\
13--16 	& Die Dame dreht unter dem rechten Arm des Herren hindurch zurück auf die Kreisbahn und es wird wieder durchgefasst.\\
\danceeasymarker
\danceinstructionsend


%%%%%%%%%%%%%%%%%%%%%%%%%%%%%%%%%%%%%%%%%%%%%%%%%%%%%%%%%%%%%%%%%%%%%%%%%%%%%%%%
% Duke of Kent's Waltz
%%%%%%%%%%%%%%%%%%%%%%%%%%%%%%%%%%%%%%%%%%%%%%%%%%%%%%%%%%%%%%%%%%%%%%%%%%%%%%%%
\newpage

\dancename{Duke of Kent's Waltz}
\iffalse\subsection{Duke of Kent's Waltz}\fi % fool texstudio into displaying subsections
\danceinfo{Longway for as many as will,\\¾ Takt, Walzerschritte}
\danceinstructionsbegin
1 -- 12 & Rechthand-Stern\\
13--24& Linkhand-Stern zurück\\
\danceinstructionsel
1 -- 6 	& Paar 1 walzt in Doppelhandfassung 2 Walzerschritte abwärts\\
7 -- 12 & Paar 1 walzt in Doppelhandfassung 2 Walzerschritte zurück auf ihren Platz\\
1 -- 12 & Paar 1 wendet auf die Plätze von Paar 2 aus, auf 7--12 rückt P2 auf.\\
\danceinstructionsel
& \textit{Ab jetzt machen die Bäumchen alles mit!}\\
1 -- 12 & Paare geben sich die rechten Hände: Balancé zueinander, auseinander, Platzwechsel.\\
13--24 	& Paare geben sich die linken Hände: Balancé zueinander, auseinander, Platzwechsel.\\
\danceinstructionsel
1 -- 12 & Rechte Handtour mit der Person diagonal rechts (ja, das kann setfremd sein)\\
13--24 	& Linke Handtour mit dem Partner\\
\dancemediummarker
\danceinstructionsend


%%%%%%%%%%%%%%%%%%%%%%%%%%%%%%%%%%%%%%%%%%%%%%%%%%%%%%%%%%%%%%%%%%%%%%%%%%%%%%%%
% Emperor of the Moon
%%%%%%%%%%%%%%%%%%%%%%%%%%%%%%%%%%%%%%%%%%%%%%%%%%%%%%%%%%%%%%%%%%%%%%%%%%%%%%%%
\newpage

\dancename{Emperor of the Moon}
\iffalse\subsection{Emperor of the Moon}\fi % fool texstudio into displaying subsections
\danceinfo{Longway for as many as will}%\\wird schneller}
\danceinstructionsbegin
1--8 	& Set \& Turn links, beim Turn weit zurücklaufen\\
9--12 	& Doppel vorwärts man trifft sich wieder in der Mitte\\
13--16 	& Set rechts\\
\danceinstructionsel
1--4 	& Paar 1 wendet auf Plätze von Paar 2 aus\\
5--8 	& Paar 2 kreuzt auf Plätze von Paar1, Paar 1 rückt auf Plätze von Paar 2, Doppelhandfassung aufnehmen\\
1--8 	& Beide Paare machen eine ganze Ronde(1er innen), die 2er werden bei 5--8 auf ihre neun Plätze „geworfen“\\
\dancemediummarker
\danceinstructionsend
%\textit{Anmerkung: \\Die ganze Bewegung ist flüssig. Die 2er Kreuzen auf die 1er Plätze und gehen in genau dieser Bewegung um die 1er herum (Ronde) und immer noch in der selben Bewegung auf ihre neuen Plätze}


%%%%%%%%%%%%%%%%%%%%%%%%%%%%%%%%%%%%%%%%%%%%%%%%%%%%%%%%%%%%%%%%%%%%%%%%%%%%%%%%
% Folia (Paartanz)
%%%%%%%%%%%%%%%%%%%%%%%%%%%%%%%%%%%%%%%%%%%%%%%%%%%%%%%%%%%%%%%%%%%%%%%%%%%%%%%%
\newpage

\dancename{Folia (Paartanz)}
\iffalse\subsection{Folia (Paartanz)}\fi % fool texstudio into displaying subsections
\danceinfo{Paartanz, ¾ Takt}
\danceinstructionsbegin
\textit{Walzerhaltung.}&\\
1--3 	& Gewichtsverlagerung auf rechten (H, D links) Fuß, Zwei Taps mit dem anderen Fuß\\
4--6 	& Gewichtsverlagerung auf linken (H, D rechts) Fuß, Zwei Taps mit dem anderen Fuß\\
7--9 	& Walzerschritt rechts vor, links diagonal, rechts ran.\\
10--12 	& Walzerschritt links zurück, rechts diagonal, dann den linken Fuß hinter dem rechten kreuzen, leicht aufdrehen.\\

\danceinstructionsend
\textit{Es können Figuren aus dem Walzer und Diskofox eingefügt werden. Schön sind auch einfach Playford Figuren:}
%\textbf{Mögliche Figuren:}
\begin{multicols}{2}
\begin{itemize}
	\item Dos-à-dos
	\item Dos-à-dos mit Drehung
	\item (Hohe) Handtour
	\item (Cecil Sharp) Siding
	\item Gypsy
	\item Promenade
	\item Herren/Damen-drehung
	\item Abknien / Umrunden
	\item Platzwechsel (Fenster)
	\item Balancé / Schulterbalancé
	\item ~
	\item ~
	\item ~
	\item ~
	\item ~
	\item ~
\end{itemize}
\dancemediummarker
\end{multicols}


%%%%%%%%%%%%%%%%%%%%%%%%%%%%%%%%%%%%%%%%%%%%%%%%%%%%%%%%%%%%%%%%%%%%%%%%%%%%%%%%
% Fourpence Ha'penny Farthing
%%%%%%%%%%%%%%%%%%%%%%%%%%%%%%%%%%%%%%%%%%%%%%%%%%%%%%%%%%%%%%%%%%%%%%%%%%%%%%%%
\newpage

\dancename{Fourpence Ha'penny Farthing}
\iffalse\subsection{Fourpence Ha'penny Farthing}\fi % fool texstudio into displaying subsections
\danceinfo{~\\Longway for as many as will}
\danceinstructionsbegin
%\danceinstructionsel
1--8 	& FC Setting steps links, rechts, Doppel zurück auf Platz\\
9--16 	& FC Ronde\\
1--8 	& SC Setting steps links, rechts, Doppel zurück auf Platz\\
9--16 	& SC Ronde\\
\danceinstructionsel
1--4 	& FC Platzwechsel (schnell)\\
5--8 	& SC Platzwechsel (schnell)\\
\danceinstructionsel
1--8 	& Paar 1 Half-Figure Eight durch Paar 2 (durch Paar kreuzen)\\
1--8 	& Paar 2 Half-Figure Eight durch Paar 1 (durch Paar kreuzen)\\
1--8 	& Alle Ronde\\
\dancemediummarker
\danceinstructionsend


%%%%%%%%%%%%%%%%%%%%%%%%%%%%%%%%%%%%%%%%%%%%%%%%%%%%%%%%%%%%%%%%%%%%%%%%%%%%%%%%
% Gallopede
%%%%%%%%%%%%%%%%%%%%%%%%%%%%%%%%%%%%%%%%%%%%%%%%%%%%%%%%%%%%%%%%%%%%%%%%%%%%%%%%
\newpage

\dancename{Gallopede}
\iffalse\subsection{Gallopede}\fi % fool texstudio into displaying subsections
\danceinfo{Longway for as many as will}
\danceinstructionsbegin
1--8 	& Ganze rechte Handtour\\
1--8 	& Ganze linke Handtour\\
1--8 	& Kreuzfassung linke Tour\\
1--8 	& Dos-à-Dos\\
\danceinstructionsel
1--4 	& Paar 1 Seitgalopp runter\\
5--8 	& Paar 1 Seitgalopp zurück\\
1--8 	& Paar 1 wendet an das Ende der Gasse aus, alle anderen schließen sich in einer Polonaise an\\
1--8 	& Paar 1 bildet am Ende der Gasse ein Tor, alle anderen gehen hindurch, es gibt ein neues Paar 1\\
\danceeasymarker
\danceinstructionsend


%%%%%%%%%%%%%%%%%%%%%%%%%%%%%%%%%%%%%%%%%%%%%%%%%%%%%%%%%%%%%%%%%%%%%%%%%%%%%%%%
% Gathering Peascods
%%%%%%%%%%%%%%%%%%%%%%%%%%%%%%%%%%%%%%%%%%%%%%%%%%%%%%%%%%%%%%%%%%%%%%%%%%%%%%%%
\newpage

\dancename{Gathering Peascods}
\iffalse\subsection{Gathering Peascods}\fi % fool texstudio into displaying subsections
\danceinfo{Kreistanz, 3 Paare, Herr/Dame}
\danceinstructionsbegin
1 -- 8 	& Durchgefasst acht Schritte links herum im Kreis\\
9 -- 12 & Turn left\\
1 -- 8 	& Acht Schritte rechts herum im Kreis\\
9 -- 12 & Turn right\\
\danceinstructionsel
& \textbf{Herren Laufen}\\
1 -- 12 & Die Herren treten in die Kreismitte, gehen einmal links herum (UZS) und drehen über links auf ihren Platz zurück\\
\danceinstructionsel
& \textbf{Damen Laufen}\\
1 -- 12 & Die Damen treten in die Kreismitte, gehen einmal rechts herum (GUZS) und drehen über rechts auf ihren Platz zurück\\
\danceinstructionsel
& \textbf{Herren Klatschen}\\
1 -- 4 	& Herren Double vor in Mitte auf 4 Klatschen\\
5 -- 8 	& Herren Double zurück, Damen vor und auf 4 Klatschen\\
9 -- 12 & Damen Double zurück, Herren vor und auf 4 Klatschen\\
13--16 	& Herren drehen über (links) auf ihren Platz zurück\\
\danceinstructionsel
& \textbf{Damen Klatschen}\\
1 -- 4 	& Damen Double vor in Mitte auf 4 Klatschen\\
5 -- 8 	& Damen Double zurück, Herren vor und auf 4 Klatschen\\
9 -- 12 & Herren Double zurück, Damen vor und auf 4 Klatschen\\
13--16 	& Damen drehen über (rechts) auf ihren Platz zurück\\
\danceinstructionsel
\danceinstructionsel
\danceinstructionsel
\danceinstructionsel
\textbf{Strophe} & \textbf{II}\\
1 -- 8 	& Siding links mit Partner\\
9 -- 12 & Turn links\\
1 -- 8 	& Siding rechts mit Partner\\
9 -- 12 & Turn rechts\\
\danceinstructionsel
1 -- 12 & \textbf{Damen Laufen links}\\
1 -- 12 & \textbf{Herren Laufen rechts}\\
1 -- 16	& \textbf{Damen Klatschen (links)}\\
1 -- 16 & \textbf{Herren Klatschen (rechts)}\\
\danceinstructionsel
\danceinstructionsel
\danceinstructionsel
\danceinstructionsel
\textbf{Strophe} & \textbf{III}\\
1 -- 8 	& Handtour links mit Partner\\
9 -- 12 & Turn links\\
1 -- 8 	& Handtour rechts mit Partner\\
9 -- 12 & Turn rechts\\
\danceinstructionsel
1 -- 12 & \textbf{Herren Laufen links}\\
1 -- 12 & \textbf{Damen Laufen rechts}\\
1 -- 16	& \textbf{Herren Klatschen (links)}\\
1 -- 16 & \textbf{Damen Klatschen (rechts)}\\
\dancemediummarker
\danceinstructionsend


%%%%%%%%%%%%%%%%%%%%%%%%%%%%%%%%%%%%%%%%%%%%%%%%%%%%%%%%%%%%%%%%%%%%%%%%%%%%%%%%
% Grimstock
%%%%%%%%%%%%%%%%%%%%%%%%%%%%%%%%%%%%%%%%%%%%%%%%%%%%%%%%%%%%%%%%%%%%%%%%%%%%%%%%
\newpage

\dancename{Grimstock}
\iffalse\subsection{Grimstock}\fi % fool texstudio into displaying subsections
\danceinfo{Longway for three couples}
\danceinstructionsbegin
%\textbf{Strophe}&\textbf{I}\\
1 -- 8 	& Lead up \& down\\
9 -- 16 & Set \& Turn links\\
1 -- 16 	& Wiederholung rechts\\
\danceinstructionsel
\textbf{Refrain}&\\
1 -- 16 & Herren und Damen tanzen auf den Seitenlinien eine Hecke. Dabei startet Paar 1 in die Gassenmitte.\\
\danceinstructionsel
%\textbf{Strophe}&\textbf{II}\\
1 -- 8 	& Siding links\\
9 -- 16 	& Set \& Turn links\\
1 -- 8 	& Wiederholung rechts\\
\danceinstructionsel
\danceinstructionsel
\textbf{Refrain}&\\
1 -- 16 & Wie auch schon vorher. Diesmal bilden die gerade außen gehenden Pärchen Tore, durch die die inneren durchgehen.\\
\danceinstructionsel
%\textbf{Strophe}&\textbf{III}\\
1 -- 8 	& Armtour links\\
9 -- 16 	& Set \& Turn links\\
1 -- 16 & Wiederholung rechts\\
\danceinstructionsel
\textbf{Refrain}&\\
1 -- 16 & Wie auch schon vorher (ohne Tore), Paar 1 kreuzt aber jedes mal, wenn es Paar 2 passiert\\
\danceinstructionsend
\textit{(Keine Wiederholungen)}
\dancedifficultmarker


%%%%%%%%%%%%%%%%%%%%%%%%%%%%%%%%%%%%%%%%%%%%%%%%%%%%%%%%%%%%%%%%%%%%%%%%%%%%%%%%
% Heptathlon
%%%%%%%%%%%%%%%%%%%%%%%%%%%%%%%%%%%%%%%%%%%%%%%%%%%%%%%%%%%%%%%%%%%%%%%%%%%%%%%%
\newpage

\dancename{Heptathlon}
\iffalse \subsection{Heptathlon}\fi % fool texstudio into displaying subsections
\danceinfo{7 Tänzer, in einem 'H' aufgestellt}
\danceinstructionsbegin
1 -- 8 	& Stern zu dritt oben rechts\\
1 -- 8 	& Stern zu dritt oben links\\
1 -- 8 	& Stern zu dritt unten rechts\\
1 -- 8 	& Stern zu dritt unten links\\
\danceinstructionsel
1 -- 16 & Heckenacht der mittleren Reihe\\
1 -- 8 	& Die Mitte tanzt mit rechts oben 1½ Ronden\\
~		& \textit{bzw. etwas beschwingtes, das in einem Platzwechsel resultiert}\\
1 -- 8 	& Set \& Turn links auf die nächste Außenposition\\
\danceeasymarker
\danceinstructionsend


%%%%%%%%%%%%%%%%%%%%%%%%%%%%%%%%%%%%%%%%%%%%%%%%%%%%%%%%%%%%%%%%%%%%%%%%%%%%%%%%
% Hey Boys Up Go We
%%%%%%%%%%%%%%%%%%%%%%%%%%%%%%%%%%%%%%%%%%%%%%%%%%%%%%%%%%%%%%%%%%%%%%%%%%%%%%%%
\newpage

\dancename{Hey Boys Up Go We}
\iffalse\subsection{Hey Boys Up Go We}\fi % fool texstudio into displaying subsections
\danceinfo{Longway for as many as will}
\danceinstructionsbegin
1 -- 8 	& Paar 1 \& Dame 2 Ronde zu dritt\\
1 -- 8 	& Paar 1 \& Herr 2 Ronde zu dritt\\
1 -- 8 	& Fall back \& meet\\
1 -- 8 	& Dos-à-dos auf der Seitenlinie\\
\danceinstructionsel
1 -- 4 	& Paar 1 klatscht: Eigene Hände, rechte Hände, eigene Hände, linke Hände\\
5 -- 8 	& Paar 1 cast down, Paar 2 lead up\\
\danceinstructionsel
1 -- 4 	& Herr 1 \& 2 klatschen: Eigene Hände, rechte, eigene, linke\\
5 -- 8 	& Dame 1 \& 2 klatschen: Eigene Hände 2x, Beide Hände 2x\\
\dancemediummarker
\danceinstructionsend


%%%%%%%%%%%%%%%%%%%%%%%%%%%%%%%%%%%%%%%%%%%%%%%%%%%%%%%%%%%%%%%%%%%%%%%%%%%%%%%%
% Hole in the Wall
%%%%%%%%%%%%%%%%%%%%%%%%%%%%%%%%%%%%%%%%%%%%%%%%%%%%%%%%%%%%%%%%%%%%%%%%%%%%%%%%
\newpage

\dancename{Hole in the Wall}
\iffalse\subsection{Hole in the Wall}\fi % fool texstudio into displaying subsections
\danceinfo{Longway for as many as will\\¾ Takt, Walzerschritte}
\danceinstructionsbegin
1--12 	& Paar 1 Cast down \& Lead up\\
1--12 	& Paar 2 Cast up \& Lead down\\
~ & ~ \\
1--6 	& FC Platzwechsel rechte Hand, kurz in der Mitte verharren\\
7--12	& SC Platzwechsel linke Hand, kurz in der Mitte verharren\\
~ & ~ \\
1--6 	& Halber Setkreis\\
7--12	& Paar 1 Cast down und zieht Paar 2 hoch \\
\danceeasymarker
\danceinstructionsend


%%%%%%%%%%%%%%%%%%%%%%%%%%%%%%%%%%%%%%%%%%%%%%%%%%%%%%%%%%%%%%%%%%%%%%%%%%%%%%%%
% Hunt the Squirrel
%%%%%%%%%%%%%%%%%%%%%%%%%%%%%%%%%%%%%%%%%%%%%%%%%%%%%%%%%%%%%%%%%%%%%%%%%%%%%%%%
\newpage

\iffalse\subsection{Hunt the Squirrel}\fi % fool texstudio into displaying subsections
\dancename{Hunt the Squirrel}
\origininfo{Playford, abgewandelt}{Playford}{}
\danceinfo{Longway for as many as will}
\danceinstructionsbegin
1 -- 8 & P1 Lead Down \& Cast up\\
9 -- 16 & H1/H2 Lead durch die Damen und Cast zurück\\
17--24 & P2 Lead Up \& Cast down\\
25--32 & D1/D2 Lead durch Herren und Cast zurück\\
\danceinstructionsel
1 -- 8 & FC Setting steps (links) zueinander \& Turn links zurück\\
9 -- 16 & SC Setting steps (links) zueinander \& Turn links zurück\\
\danceinstructionsel
1 -- 4 & Halber Setkreis \\
5 -- 8 & Linien Fall back\\
9 -- 12 & Setting steps aufeinander zu\\
13 -- 16 & Paare halbe Ronde
\dancedifficultmarker
\danceinstructionsend


%%%%%%%%%%%%%%%%%%%%%%%%%%%%%%%%%%%%%%%%%%%%%%%%%%%%%%%%%%%%%%%%%%%%%%%%%%%%%%%%
%	I care not for these Ladies
%%%%%%%%%%%%%%%%%%%%%%%%%%%%%%%%%%%%%%%%%%%%%%%%%%%%%%%%%%%%%%%%%%%%%%%%%%%%%%%%
\newpage

\dancename{I care not for these Ladies}
\iffalse\subsection{I care not for these Ladies}\fi % fool texstudio into displaying subsections
\danceinfo{Kreistanz\\Herr/Dame durchgefasst}
\danceinstructionsbegin
\textbf{Teil} & \textbf{I}\\
1 -- 8 & Gehüpfte Chassés links herum im Kreis\\
9 -- 16 & Gehüpfte Chassés rechts herum zurück\\
1 -- 8 & Set \& Turn links mit dem Partner\\
\danceinstructionsel
\textbf{Refrain:} & (bzw. Fortschritt)\\
1 -- 4 & Halbe rechte Handtour mit dem Partner\\
5 -- 8 & dann weiter auf Kreisbahn mit dem entgegenkommenden neuen Partner eine halbe linke Handtour\\
9 -- 16 & Mit dem nun entgegenkommenden Partner eine anderthalbe Ronde, sodass man wieder richtig im Kreis steht.\\
\danceinstructionsel
\textbf{Teil} & \textbf{II}\\
1 -- 8 & Die neuen Paare Siding links\\
9 -- 16 & Siding rechts \\
1 -- 8 & Set \& Turn links (manchmal auch rechts für den Ausgleich)\\
\danceinstructionsel
\textbf{Refrain} & \textbf{wiederholen}\\
\danceinstructionsel
\textbf{Teil} & \textbf{III}\\
1 -- 8 & Die neuen Paare Linke Armtour (alternativ Arme einhaken)\\
9 -- 16 & Rechte Armtour (oder Alternative)\\
1 -- 8 & Set \& Turn links\\

\danceinstructionsel
\textbf{Refrain} & \textbf{wiederholen}
\danceeasymarker
\danceinstructionsend


%%%%%%%%%%%%%%%%%%%%%%%%%%%%%%%%%%%%%%%%%%%%%%%%%%%%%%%%%%%%%%%%%%%%%%%%%%%%%%%%
%	Indian Queen
%%%%%%%%%%%%%%%%%%%%%%%%%%%%%%%%%%%%%%%%%%%%%%%%%%%%%%%%%%%%%%%%%%%%%%%%%%%%%%%%
\newpage

\dancename{Indian Queen}
\iffalse\subsection{Indian Queen}\fi % fool texstudio into displaying subsections
\danceinfo{Longway for as many as will}

\danceinstructionsbegin
1 -- 4  & First Corner Set zum Partner dann zum Kontra\\
5 -- 8  & First Corner Turn rechts\\
9 -- 16 & First Corner Ronde\\
1 -- 8  & Second Corner Set \& Turn zum Partner, dann Kontra\\
9 -- 16 & Second Corner Ronde\\
\danceinstructionsel
1 -- 8  & Mühle rechts herum (rechte Hände) auf 8 Klatschen\\
9 -- 16 & Mühle links herum zurück, auf 8 Klatschen\\
\danceinstructionsel
1 -- 8  & Dos-à-dos\\
1 -- 8  & $\nicefrac{3}{4}$ Kette\\
\danceeasymarker
\danceinstructionsend

%%%%%%%%%%%%%%%%%%%%%%%%%%%%%%%%%%%%%%%%%%%%%%%%%%%%%%%%%%%%%%%%%%%%%%%%%%%%%%%%
% Jamaica
%%%%%%%%%%%%%%%%%%%%%%%%%%%%%%%%%%%%%%%%%%%%%%%%%%%%%%%%%%%%%%%%%%%%%%%%%%%%%%%%
\newpage

\dancename{Jamaica}
\iffalse\subsection{Jamaica}\fi % fool texstudio into displaying subsections
\danceinfo{Longway, Doppelter Fortschritt}
\danceinstructionsbegin
\textbf{I}& \\
1--4 	& P1 „Rechte Hand und linke Hand“(geben)\\
5--8 	& P1 „Auf die andere Seite“ (Platzwechsel)\\
9--16 	& H1+D2 / H2+D1 „Rechte Hand und linke Hand“ + Platzwechsel\\
%\danceinstructionsel
1--12 	& P1 Figure of Eight um P2 \\
13--16	& P1 halbe Ronde (ergibt sich aus Figure of Eight)\\
& \textit{Erster Fortschritt (Es gibt neue Sets)}\\
\textbf{II} & \\
1--8 	& FC Ronde\\
9--16	& SC Ronde\\
1--8 	& H1+H2, D1+D2  1½ Ronde\\
9--16 	& Dos-á-Dos
\dancemediummarker
\danceinstructionsend


%%%%%%%%%%%%%%%%%%%%%%%%%%%%%%%%%%%%%%%%%%%%%%%%%%%%%%%%%%%%%%%%%%%%%%%%%%%%%%%%
% Jenny Pluck Pears
%%%%%%%%%%%%%%%%%%%%%%%%%%%%%%%%%%%%%%%%%%%%%%%%%%%%%%%%%%%%%%%%%%%%%%%%%%%%%%%%
\newpage

\dancename{Jenny Pluck Pears}
\iffalse\subsection{Jenny Pluck Pears}\fi % fool texstudio into displaying subsections
\danceinfo{Kreistanz, 3 Paare, durchgefasst}
\danceinstructionsbegin
\textbf{Strophe}&\textbf{I}\\
1 -- 8 	& Acht Schritte nach links gehen\\
9 -- 16 & Jeder ein Set \& Turn links\\
1 -- 8 	& Acht Schritte nach rechts gehen\\
9 -- 16 & Jeder ein Set \& Turn rechts\\
\danceinstructionsel
\textbf{Refrain}&\textbf{(Herren Dominant)}\\
1 -- 4 	& Herr 1 dreht seine Dame an der rechten Hand in den Kreis\\
5 -- 8 	& Herr 2 ebenso\\
9 -- 12 & Herr 3 ebenso\\
13 -- 16& Alle eine Referenz\\
1 -- 8 	& Die Herren gehen 8 Schritte vor den Damen nach links\\
9 -- 16 & Die Herren tanzen ein Set \& Turn links\\
1 -- 8 	& Die Herren gehen 8 Schritte vor den Damen nach rechts\\
9 -- 16 & Die Herren tanzen ein Set \& Turn rechts\\
1 -- 16 & Herr 1 - 3 dreht seine Dame aus der Mitte \& Referenz\\
\danceinstructionsel
\textbf{Strophe}&\textbf{II}\\
1 -- 8 	& Paare Siding links\\
9 -- 16 & Set \& Turn links\\
1 -- 8 	& Paare Siding rechts\\
9 -- 16 & Set \& Turn rechts\\
\danceinstructionsel
\textbf{Refrain}&\textbf{(Damen Dominant)}\\
&\textit{ergo ersetze Dame mit Herr und andersherum}\\
\danceinstructionsel
\danceinstructionsel
\textbf{Strophe}&\textbf{III}\\
1 -- 8 	& Paare Armtour links\\
9 -- 16 & Set \& Turn links\\
1 -- 8 	& Paare Armtour rechts\\
9 -- 16 & Set \& Turn rechts\\
\danceinstructionsel
\textbf{Refrain}&\textbf{(Herren Dominant)}\\
\dancemediummarker
\danceinstructionsend


%%%%%%%%%%%%%%%%%%%%%%%%%%%%%%%%%%%%%%%%%%%%%%%%%%%%%%%%%%%%%%%%%%%%%%%%%%%%%%%%
%	Korobushka (Gasse)
%%%%%%%%%%%%%%%%%%%%%%%%%%%%%%%%%%%%%%%%%%%%%%%%%%%%%%%%%%%%%%%%%%%%%%%%%%%%%%%%
\newpage

\dancename{Korobushka (Gasse)}
\iffalse\subsection{Korobushka (Gasse)}\fi % fool texstudio into displaying subsections
\danceinfo{Longway for as many as will\\(ohne Fortschritt)}
%\dancemediummarker
\textit{\footnotesize Korobushka-Fassung: die Paare geben sich jeweils die rechten und linken Hände und halten diese gekreuzt.}\\
Paare in Korobushka-Fassung nach oben schauend.
\danceinstructionsbegin
1 -- 4  & Double vor (letzter Tap)\\
5 -- 8  & Double zurück (letzter Tap)\\
9 -- 16 & Wiederholen, am Ende zueinander drehen (ohne Fassung zu lösen)\\
1 -- 2  & Zueinander hüpfen\\
3 -- 4 	& Auseinander hüpfen\\
5 -- 8  & Nocheinmal Zueinander \& auseinander\\
9 -- 12  & Platzwechsel (danach Hände loslassen)\\
13 -- 16 & Dreimal Stampfen und Klatschen (16 ohne Klatschen)\\
1 -- 4 & Nach links mit rechts kreuzen nach links und Kick\\
5 -- 8 & Nach rechts mit links kreuzen nach rechts und Kick\\
9 -- 12 & (Korobushkafassung) Zueinander \& Auseinander\\
13 -- 16 & Platzwechsel
\dancemediummarker
\danceinstructionsend


%%%%%%%%%%%%%%%%%%%%%%%%%%%%%%%%%%%%%%%%%%%%%%%%%%%%%%%%%%%%%%%%%%%%%%%%%%%%%%%%
% Mulberry Garden
%%%%%%%%%%%%%%%%%%%%%%%%%%%%%%%%%%%%%%%%%%%%%%%%%%%%%%%%%%%%%%%%%%%%%%%%%%%%%%%%
\newpage

\dancename{Mulberry Garden}
\iffalse\subsection{Mulberry Garden}\fi % fool texstudio into displaying subsections
\danceinfo{Longway for as many as will}
\danceinstructionsbegin
1--8 	& Doppel vor und zurück\\
9--16 	& Wiederholung\\
\danceinstructionsel
1--8 	& Herren und Damenreihen Fall back and Meet\\
1--8 	& Partner Ronde\\
\danceinstructionsel
1--8 	& Dos-a-Dos mit dem Partner\\
1--8 	& Dos-a-Dos auf der Linie\\
\danceinstructionsel
1--4 	& halber Setkreis\\
5--8 	& halbe Ronde mit Partner\\
1--8 	& Fontäne\\
\dancemediummarker
\danceinstructionsend


%%%%%%%%%%%%%%%%%%%%%%%%%%%%%%%%%%%%%%%%%%%%%%%%%%%%%%%%%%%%%%%%%%%%%%%%%%%%%%%%
% Newcastle Circle
%%%%%%%%%%%%%%%%%%%%%%%%%%%%%%%%%%%%%%%%%%%%%%%%%%%%%%%%%%%%%%%%%%%%%%%%%%%%%%%%
\newpage

\dancename{Newcastle Circle}
%\origininfo{Abwandlung des Newcastle für mehr als vier Paare}{arg2}{arg3}
\iffalse \subsection{Newcastle Circle}\fi % fool texstudio into displaying subsections
\danceinfo{Kreistanz, Herr/Dame durchgefasst}
\danceinstructionsbegin
1 -- 8 & Meet \& Fall back aller in die Mitte\\
9 -- 16 & Alle Set \& Turn links\\
1 -- 16 & Wiederholen(rechts)\\
\danceinstructionsel
1 -- 8 & Handtour links mit Partner \\
9 -- 16 & Handtour rechts mit anderem Partner \\
\danceinstructionsel
1 -- 8 & Dos-a-dos der Partner \\
9 -- 12 & Partner passieren sich rechtsschultrig \\
13 -- 16 & Schnelle Ronde mit entgegenkommendem neuen Partner \\
\danceeasymarker
\danceinstructionsend


%%%%%%%%%%%%%%%%%%%%%%%%%%%%%%%%%%%%%%%%%%%%%%%%%%%%%%%%%%%%%%%%%%%%%%%%%%%%%%%%
% Old Bachelor
%%%%%%%%%%%%%%%%%%%%%%%%%%%%%%%%%%%%%%%%%%%%%%%%%%%%%%%%%%%%%%%%%%%%%%%%%%%%%%%%
\newpage

\dancename{Old Bachelor}
\iffalse\subsection{Old Bachelor}\fi % fool texstudio into displaying subsections
\danceinfo{Longway for as many as will}
%\dancedifficultmarker
\danceinstructionsbegin
1 -- 8 	& P1 passiert sich rechtschultrig, geht um den jeweiligen Contra herum und trifft sich zwischen und mit P2 zu einer Line of Four \\
9 -- 16 	& Lead up and back der Line of Four, am Ende zum Kontra wenden\\
\danceinstructionsel
1 -- 16 	& H1 D2/H2 D1 siding rechtsschultrig\\
5 -- 8 	& Dann zurück linksschultrig(nur P1)\\
9 -- 12 & Wieder H1 D2/H2 D1 siding linksschultrig\\
13 -- 16& zurück und P1 wendet sich zueinander\\
\danceinstructionsel
\danceinstructionsel
\danceinstructionsel
1 -- 8	& P1 tanzt eine 1¼ Ronde, Herr schaut am Ende nach oben.\\
9 -- 12 	& P1 tanzt mit D2 in einem 3er Kreis halb rechtsherum\\
13 -- 16 	& H1 und Damen Turn rechts\\
\danceinstructionsel
1 -- 4 	& Damen und H2 tanzen im 3er Kreis halb rechtsherum, H2 steht dann auf der Damenseite. \\
5 -- 8 	& Turn links auf Linie. Paare nun improper\\
9 -- 16 	& 2er Kette\\
\dancedifficultmarker
\danceinstructionsend


%%%%%%%%%%%%%%%%%%%%%%%%%%%%%%%%%%%%%%%%%%%%%%%%%%%%%%%%%%%%%%%%%%%%%%%%%%%%%%%%
% Pavane d'Honneur
%%%%%%%%%%%%%%%%%%%%%%%%%%%%%%%%%%%%%%%%%%%%%%%%%%%%%%%%%%%%%%%%%%%%%%%%%%%%%%%%
\newpage

\dancename{Pavane d'Honneur}
\iffalse\subsection{Pavane d'Honneur}\fi % fool texstudio into displaying subsections
\danceinfo{Reihentanz, versetzte Aufstellung}
\danceinstructionsbegin
\textbf{I}&\\
1 -- 8 	& Simple schräg vorwärts links, rechts, links, rechts\\
9 -- 16 	& Simple schräg zurück links, rechts, links, rechts\\
\danceinstructionsel
\textbf{II}&\\
1 -- 8 	& Die rechten Paare vier Simple nach links, Die Linken vier rechts, auf zwei in Flucht stehen, auf vier getauscht haben\\
9 -- 16  	& Wiederholung zurück\\
\danceinstructionsel
\textbf{III}&\\
1 -- 16 	& Der Herr kniet ab, Die Dame umrundet ihn in Achteln\\
1 -- 16 	& Der Herr erhebt sich. Der Herr umrundet die Dame in Achteln, dabei dreht sie sich mit
\dancemediummarker
\danceinstructionsend


%%%%%%%%%%%%%%%%%%%%%%%%%%%%%%%%%%%%%%%%%%%%%%%%%%%%%%%%%%%%%%%%%%%%%%%%%%%%%%%%
% Pavane La Battaglia
%%%%%%%%%%%%%%%%%%%%%%%%%%%%%%%%%%%%%%%%%%%%%%%%%%%%%%%%%%%%%%%%%%%%%%%%%%%%%%%%
\newpage

\dancename{Pavane La Battaglia}
\iffalse\subsection{Pavane La Battaglia}\fi % fool texstudio into displaying subsections
\danceinfo{Reihentanz, gerade Anzahl Paare}
%\dancedifficultmarker
\danceinstructionsbegin
\textbf{I}&\\
1 -- 8 		& Alle tanzen ein Simple links, Simple rechts, Double links\\
9 -- 16		& Alle tanzen ein Simple rechts, Simple links, Double rechts\\
1 -- 16		& Wiederholen \& Am Ende leicht auffächern\\
\danceinstructionsel
\textbf{IIa}&\\
1 -- 4 		& Paar 1 geht seitwärts zwei Simple auseinander\\
& Paar 2 geht seitwärts zwei Simple zueiander\\
5 -- 8 		& Paar 1 tanzt ein Double rückwärts, Paar 2 vorwärts\\
1 -- 4 		& Paar 1 seitwärts zueinander, Paar 2 auseinander\\
5 -- 8 		& Paar 1 tanzt ein Double vorwärts, Paar 2 rückwärts\\
\textbf{IIb}&\\
1 -- 4 		& Paar 1 tanzt ein Double rückwärts, Paar 2 vorwärts\\
5 -- 8 		& Paar 1 geht seitwärts zwei Simple auseinander\\
1 -- 4 		& Paar 1 tanzt ein Double vorwärts, Paar 2 rückwärts\\
5 -- 8 		& Paar 1 seitwärts zueinander, Paar 2 auseinander\\
\danceinstructionsel
\textbf{III} & \textit{Die Paare wenden sich zueinander}\\
1 -- 4 		& Alle tanzen ein Simple links, Simple rechts\\
5 -- 8 		& Drehung auf dem Platz mit einem Double links\\
9 -- 12 	& Alle tanzen ein Simple rechts, Simple links\\
13 -- 16 	& Halbe Drehung mit dem Partner mit einem Double rechts\\
\danceinstructionsel
1 -- 16 	& Wiederholung\\
\danceinstructionsel
\textbf{IV} & \textit{Die Paare wenden sich wieder nach vorne}\\
1 -- 2 		& Pause\\
3 -- 8 		& Paar 1 wendet in 3 Simples aus, Paar 2 rückt auf.\\
1 -- 2 		& Pause\\
3 -- 8 		& Paar 2 wendet in 3 Simples aus, Paar 1 rückt auf.\\
& Alle wenden sich nach unten\\
1 -- 2 		& Pause\\
3 -- 8 		& Paar 2 wendet in 3 Simples aus, Paar 1 rückt auf.\\
1 -- 2 		& Pause\\
3 -- 8 		& Paar 1 wendet in 3 Simples aus, Paar 2 rückt auf.\\
& Alle wenden sich nach oben\\
\textbf{V} 	& \\
1 -- 4 		& Alle tanzen ein Simple links, Simple rechts\\
5 -- 8 		& Jedes Paar dreht sich gemeinsam eine Vierteldrehung\\
& Paar 1 um den Herren, Paar 2 um die Dame\\
9 -- 16 	& Wiederholung, am Ende zur Viererkette durchfassen\\
\danceinstructionsel
\textbf{VI} & \\
1 -- 8 		& Alle tanzen ein Simple links, Simple rechts, Double links\\
9 -- 16		& Alle tanzen ein Simple rechts, Simple links, Double rechts\\
\danceinstructionsel
1 -- 4 		& Alle tanzen ein Simple links, Simple rechts\\
5 -- 8 		& Jedes Paar dreht sich gemeinsam eine Vierteldrehung\\
& Paar 1 um den Herren, Paar 2 um die Dame\\
9 -- 16 	& Wiederholung, am Ende zur Viererkette durchfassen\\
\danceinstructionsel
1 -- 8 		& Alle tanzen ein Simple links, Simple rechts, Double links\\
9 -- 16		& Alle tanzen ein Simple rechts, Simple links, Double rechts\\
&\textit{Der Tanz endet in einer knienden Referenz}
\dancedifficultmarker
\danceinstructionsend


%%%%%%%%%%%%%%%%%%%%%%%%%%%%%%%%%%%%%%%%%%%%%%%%%%%%%%%%%%%%%%%%%%%%%%%%%%%%%%%%
% Queens Jig
%%%%%%%%%%%%%%%%%%%%%%%%%%%%%%%%%%%%%%%%%%%%%%%%%%%%%%%%%%%%%%%%%%%%%%%%%%%%%%%%
\newpage

\dancename{Queens Jig}
\iffalse\subsection{Queens Jig}\fi % fool texstudio into displaying subsections
\danceinfo{Longway for as many as will}
\danceinstructionsbegin
1 -- 8 	& FC Siding\\
9 -- 16 & FC Set \& Turn links\\
1 -- 8 	& SC Siding\\
9 -- 16 & SC Set \& Turn rechts\\
\danceinstructionsel
1 -- 4 	& FC Platzwechsel\\
5 -- 8 	& SC Platzwechsel\\
1 -- 4 	& Balancé zurück\\
5 -- 8 	& Platzwechsel mit Partner\\
\danceinstructionsel
1 -- 12	& volle Mühle (oder vielleicht auch Setkreis)\\
13 -- 16& Turn links\\
\dancemediummarker
\danceinstructionsend


%%%%%%%%%%%%%%%%%%%%%%%%%%%%%%%%%%%%%%%%%%%%%%%%%%%%%%%%%%%%%%%%%%%%%%%%%%%%%%%%
% Red House
%%%%%%%%%%%%%%%%%%%%%%%%%%%%%%%%%%%%%%%%%%%%%%%%%%%%%%%%%%%%%%%%%%%%%%%%%%%%%%%%
\newpage

\dancename{Red House}
\iffalse\subsection{Red House}\fi % fool texstudio into displaying subsections
\danceinfo{Longway for as many as will}

\danceinstructionsbegin
1 -- 8 	& P1 Meet \& Fallback, P2 Fallback \& Meet\\
1 -- 8  	& P1 Set nach oben, unten, P1 Cast down, P2 lead up\\
1 -- 8 	& P1 Meet \& Fallback, P2 Fallback \& Meet\\
1 -- 8 	& P1 Set nach unten, oben, P1 Cast up, P2 lead down\\
\danceinstructionsel
1 -- 16 	& Herr 1 wendet aus, Dame 1 folgt. Sie gehen einmal um Paar 2 herum. Wenn sie hinter Dame 2 stehen (13) rückt Paar 2 auf und Paar 1 übernimmt den Platz.\\
1 -- 16 	& Dame 2 wendet aus, Herr 2 folgt. Sie gehen einmal um Paar 1 herum. Wenn sie hinter Herr 1 stehen (13) rückt Paar 1 auf und Paar 2 übernimmt den Platz.\\
\danceinstructionsel
\danceinstructionsel
1 -- 16 	& Herr 2 geht mit Paar 1 in eine Hecken-Acht\\
1 -- 12 	& Dame 2 geht in eine Hecken-Acht mit Paar 1 (bisschen zügiger)\\
13 -- 16 	& Paar 1 wendet aus, Paar 2 schließt auf
\dancedifficultmarker
\danceinstructionsend


%%%%%%%%%%%%%%%%%%%%%%%%%%%%%%%%%%%%%%%%%%%%%%%%%%%%%%%%%%%%%%%%%%%%%%%%%%%%%%%%
% Schiarazula Marazula
%%%%%%%%%%%%%%%%%%%%%%%%%%%%%%%%%%%%%%%%%%%%%%%%%%%%%%%%%%%%%%%%%%%%%%%%%%%%%%%%
\newpage

\dancename{Schiarazula Marazula}
\iffalse\subsection{Schiarazula Marazula}\fi % fool texstudio into displaying subsections
\danceinfo{Kreistanz, Herr/Dame durchgefasst}

\danceinstructionsbegin
1 -- 4	& Doppel nach links, kreuzend(vorne)\\
5 -- 8 	& Doppel nach rechts, kreuzend (vorne)\\
9 -- 16	& Wiederholen\\
\danceinstructionsel
%Nach innen
1 		& Alle setzen einen Fuß nach innen in den Kreis hinein, sodass der 	Fremde Partner angeschaut wird \\
2		& Dem Partner mit beiden Händen zuschnipsen.\\
3 -- 4	& Weiter in den Kreis zum eigenen Partner drehen und schnipsen\\
5 -- 6	& Weiter in den Kreis zum fremden Partner drehen und schnipsen\\
7 -- 8	& Nach außen drehen alle dreimal Klatschen (auf 6, 6\textsuperscript{und}, und 7)\\
\danceinstructionsel
%Nach außen
1 -- 2 	& Nach außen zum fremden Partner drehen und schnipsen\\
3 -- 4 	& Weiter nach außen zum eigenen Partner drehen und schnipsen\\
5 -- 6 	& Weiter nach außen zum eigenen Partner drehen und schnipsen\\
7 -- 8	& Nach innen drehen alle dreimal Klatschen (auf 6, 6\textsuperscript{und}, und 7), 	wieder durchfassen
\danceeasymarker
\danceinstructionsend


%%%%%%%%%%%%%%%%%%%%%%%%%%%%%%%%%%%%%%%%%%%%%%%%%%%%%%%%%%%%%%%%%%%%%%%%%%%%%%%%
% Seepferd und Biber
%%%%%%%%%%%%%%%%%%%%%%%%%%%%%%%%%%%%%%%%%%%%%%%%%%%%%%%%%%%%%%%%%%%%%%%%%%%%%%%%
\newpage

\dancename{Seepferd und Biber}
\iffalse\subsection{Seepferd und Biber}\fi % fool texstudio into displaying subsections
\danceinfo{Kreistanz, Herren innen, Damen außen\\gerade Anzahl an Paaren}

\textit{Anmerkung: Die Paare sind abwechselnd Seepferde und Biber\\ (Seefperdchen tanzen mit dem vom Herrn aus linken Paar,\\ Biber tanzen mit dem vom Herrn aus rechtem Paar)}
\danceinstructionsbegin

1 -- 16 &	Biber bilden Tore, Seepferde machen eine Figure of Eight hindurch\\
1 -- 16 &	Seepferde bilden Tore, Biber machen eine Figure of Eight hindurch\\
\danceinstructionsel
\textit{Anm.:} & \textit{Drehrichtungen: Biber rechts, Seepferde links}\\
\danceinstructionsel
1 -- 4	& Zweimal Klatschen, Zwei Schritte am Partner vorbei, Drehung in Tierrichtung\\
5 -- 8 	& Klatschen, Zwei Schritte am Gegenüber auf Kreisbahn vorbei, Drehung Tierrichtung\\
9 -- 12	& Klatschen, Zwei Schritte am neuen Partner vorbei, in Laufrichtung stehen bleiben\\
13 -- 16& Klatschen, zum Partner umdrehen\\
\danceinstructionsend
\textit{Neuer Partner, Der Tanz beginnt von vorne}
\danceeasymarker


%%%%%%%%%%%%%%%%%%%%%%%%%%%%%%%%%%%%%%%%%%%%%%%%%%%%%%%%%%%%%%%%%%%%%%%%%%%%%%%%
% Siege of Buda
%%%%%%%%%%%%%%%%%%%%%%%%%%%%%%%%%%%%%%%%%%%%%%%%%%%%%%%%%%%%%%%%%%%%%%%%%%%%%%%%
\newpage

\dancename{Siege of Buda}
\iffalse\subsection{Siege of Buda}\fi % fool texstudio into displaying subsections
\danceinfo{Longway for as many as will}
\danceinstructionsbegin
1 -- 8 	& Paare Dos-á-Dos\\
1 -- 8 	& Dos-á-Dos auf der Linie mit gleichgeschlechtlichem Tänzer\\
\danceinstructionsel
1 -- 4 	& Halber Einhandkreis auf der Linie\\
5 -- 8 	& Fallback im Set durchgefasst\\
9 -- 12	& Meet\\
13 -- 16& Halber Einhandkreis der Paare\\
\danceinstructionsel
1 -- 4	& Halber Setkreis\\
5 -- 8	& P1 Cast down, P2 Lead up
\dancemediummarker
\danceinstructionsend


%%%%%%%%%%%%%%%%%%%%%%%%%%%%%%%%%%%%%%%%%%%%%%%%%%%%%%%%%%%%%%%%%%%%%%%%%%%%%%%%
% Siege of Limerick
%%%%%%%%%%%%%%%%%%%%%%%%%%%%%%%%%%%%%%%%%%%%%%%%%%%%%%%%%%%%%%%%%%%%%%%%%%%%%%%%
\newpage

\dancename{Siege of Limerick}
\iffalse\subsection{Siege of Limerick}\fi % fool texstudio into displaying subsections
\danceinfo{Longway for as many as will}
\danceinstructionsbegin
1--6 	& H1 cast down, H2 rückt auf\\
7--12 	& H1 geht zwischen Damen durch, um D2 rum, und zurück auf Platz\\
1--6 	& D1 cast down, D2 rückt auf\\
7--12 	& D1 zwischen Herren durch, um H1 rum, und zurück auf Platz\\
\danceinstructionsel
1--6 	& P2 cast down, P1 lead up\\
1--6 	& Dos-à-Dos\\
1--12 	& Ganze Kette\\
\danceinstructionsel
1--6 	& P2 Handfassung zu P1 verneigen\\
7--12 	& P1 cast down, P2 lead up
\dancemediummarker
\danceinstructionsend


%%%%%%%%%%%%%%%%%%%%%%%%%%%%%%%%%%%%%%%%%%%%%%%%%%%%%%%%%%%%%%%%%%%%%%%%%%%%%%%%
% 	Siege of St. Malo
%%%%%%%%%%%%%%%%%%%%%%%%%%%%%%%%%%%%%%%%%%%%%%%%%%%%%%%%%%%%%%%%%%%%%%%%%%%%%%%%
\newpage

\dancename{Siege of St. Malo}
\iffalse\subsection{Siege of St. Malo}\fi % fool texstudio into displaying subsections
\danceinfo{Longway for as many as will}

\danceinstructionsbegin
	\danceinstructionsel
	1 -- 8  & First Corner Dos-à-dos\\
	9 -- 16 & Second Corner Dos-à-dos\\
	\danceinstructionsel
	1 -- 8  & $\nicefrac{3}{4}$ Kette\\
	\danceinstructionsel
	1 -- 8  & Set \& Turn rechts
\danceeasymarker
\danceinstructionsend


%%%%%%%%%%%%%%%%%%%%%%%%%%%%%%%%%%%%%%%%%%%%%%%%%%%%%%%%%%%%%%%%%%%%%%%%%%%%%%%%
% Tourdion
%%%%%%%%%%%%%%%%%%%%%%%%%%%%%%%%%%%%%%%%%%%%%%%%%%%%%%%%%%%%%%%%%%%%%%%%%%%%%%%%
\newpage

\dancename{Tourdion}
\iffalse\subsection{Tourdion}\fi % fool texstudio into displaying subsections
\danceinfo{Kreistanz, Herr/Dame, durchgefasst}
\danceinstructionsbegin
1--4 	& Nach links, rechts vor und zurück wiegen\\
5--16 	& 3x Wiederholen \\
\danceinstructionsel
1--2 	& Die Herren geben/heben/werfen die rechts stehende Dame einen Platz nach links, danach \\
3--4 	& Vor und zurück wiegen\\
5--16 	& 3x Wiederholen \\
\danceinstructionsel
1--16 	& Grundschritt: Links, rechts, vor, zurück\\
\danceinstructionsel
1--4 	& Die Damen geben/heben/werfen den links stehenden Herren einen Platz nach rechts, danach Vor und zurück wiegen\\
5--16 	& 3x Wiederholen
\danceeasymarker
\danceinstructionsend


%%%%%%%%%%%%%%%%%%%%%%%%%%%%%%%%%%%%%%%%%%%%%%%%%%%%%%%%%%%%%%%%%%%%%%%%%%%%%%%%
% Traubentritt
%%%%%%%%%%%%%%%%%%%%%%%%%%%%%%%%%%%%%%%%%%%%%%%%%%%%%%%%%%%%%%%%%%%%%%%%%%%%%%%%
\newpage

\dancename{Traubentritt}
\iffalse\subsection{Traubentritt}\fi % fool texstudio into displaying subsections
\danceinfo{Longway for as many as will}
\danceinstructionsbegin
1--8 	& vier Simpel abwechselnd links und rechts, auf 8 umdrehen\\
9--16 	& vier Simpel zurück – zueinander wenden\\
\danceinstructionsel
1--4	 	& Referenz Herren\\
5--8 	& Referenz Damen\\
9--12 	& Referenz Links\\
13--16 	& Referenz Rechts\\
\danceinstructionsel
1--6 	& Die Dame dreht dreimal unter der rechten Hand des Herren\\
7--8	& Referenz beider\\
9--12	& Der hinterste Herr läuft durch die Gasse an die Spitze, alle anderen Herren rücken mit zwei Anstellschritten auf
\danceeasymarker
\danceinstructionsend


%%%%%%%%%%%%%%%%%%%%%%%%%%%%%%%%%%%%%%%%%%%%%%%%%%%%%%%%%%%%%%%%%%%%%%%%%%%%%%%%
% Upon a Summers Day
%%%%%%%%%%%%%%%%%%%%%%%%%%%%%%%%%%%%%%%%%%%%%%%%%%%%%%%%%%%%%%%%%%%%%%%%%%%%%%%%
\newpage

\dancename{Upon a Summers Day}
\iffalse\subsection{Upon a Summers Day}\fi % fool texstudio into displaying subsections
\danceinfo{Longway for six}

\danceinstructionsbegin
\textbf{Strophe} & \textbf{~I}\\
1--8 	& Lead Up and Down\\
9--16 	& Set \& Turn links\\
1--16 	& Wiederholung (rechts)\\
\danceinstructionsel
\textbf{Refrain}& \\
1--8 	& Meet \& Fallback\\
9--16 	& Paar 1 nach hinten schlängeln, aufrücken\\
1--16 	& Wiederholen (Paar 2)\\
1--16 	& Wiederholen (Paar 3)\\
\danceinstructionsel
\textbf{Strophe} & \textbf{~II}\\
1--8 	& Siding links\\
9--16 	& Set \& Turn links\\
1--16 	& Wiederholen (rechts)\\
\danceinstructionsel
1-48 	& \textbf{Refrain}\\
\danceinstructionsel
\textbf{Strophe} & \textbf{~III}\\
1--8 	& Handtour links\\
9--16 	& Set \& Turn links\\
1--16 	& Wiederholung (rechts)\\
\danceinstructionsel
1--48 	& \textbf{Refrain}\\
\danceinstructionsend
\textit{(Keine Wiederholung)}
\dancemediummarker


%%%%%%%%%%%%%%%%%%%%%%%%%%%%%%%%%%%%%%%%%%%%%%%%%%%%%%%%%%%%%%%%%%%%%%%%%%%%%%%%
% Walenki
%%%%%%%%%%%%%%%%%%%%%%%%%%%%%%%%%%%%%%%%%%%%%%%%%%%%%%%%%%%%%%%%%%%%%%%%%%%%%%%%
\newpage

\dancename{Walenki}
\iffalse \subsection{Walenki}\fi % fool texstudio into displaying subsections
\danceinfo{Doppelter Kreis, Damen innen, \\Herren versetzt außen,\\ beide Kreise durchgefasst}
\danceinstructionsbegin
1 -- 8 & Die Herren gehen 8 Schritte links herum, die Damen 8 Schritte rechts\\
9 -- 16 & Die Herren 8 Schritte rechts zurück, die Damen 8 Schritte links zurück\\
\danceinstructionsel
1 -- 8 & Alle gehen gemeinsam in die Mitte und wieder zurück\\
1 -- 4 & Wieder alle in die Mitte, die Herren heben die Hände\\
5 -- 8 & Alle wieder zurück, die Herren fangen die Damen mit den Händen ein\\
1 -- 16 & Alle 8 Schritte rechts, dann 8 Schritte links\\
1 -- 8 & Alle gehen gemeinsam in die Mitte und wieder zurück\\
1 -- 8 & Alle gehen gemeinsam in die Mitte, die Herren heben die Hände wieder über die Damen zurück\\
1 -- 16 & Die Herren gehen 8 Schritte links, die Damen 8 Schritte rechts und wieder zurück\\
1 -- 8 & Tore: Die Damen heben die Hände und gehen nach außen, die Herren darunter hindurch, dann heben die Herren ihre Hände und gehen nach außen, die Damen gehen durch die Tore.\\
9 -- 16 & Wiederholung
\danceeasymarker
\danceinstructionsend


%%%%%%%%%%%%%%%%%%%%%%%%%%%%%%%%%%%%%%%%%%%%%%%%%%%%%%%%%%%%%%%%%%%%%%%%%%%%%%%%
% Woaf
%%%%%%%%%%%%%%%%%%%%%%%%%%%%%%%%%%%%%%%%%%%%%%%%%%%%%%%%%%%%%%%%%%%%%%%%%%%%%%%%
\newpage

\dancename{Woaf}
\iffalse\subsection{Woaf}\fi % fool texstudio into displaying subsections
\danceinfo{Kreistanz}

\danceinstructionsbegin
\textbf{Teil I} &\\
1 -- 4 	& Chassé links\\
5 -- 8 	& Chassé rechts\\
\danceinstructionsel
\textbf{Teil II} &\\
1 -- 2 	& Der Herr führt seine Dame nach links und blickt ihr über die rechte Schulter ins Gesicht\\
3 -- 4 	& Der Herr führt seine Dame nach rechts und blickt ihr über die linke Schulter in Gesicht\\
5 -- 8 	& Die Dame dreht sich einmal unter der erhobenen rechten Hand des Herren und nimmt dann wiede die Kiekbuschfassung ein.\\
\danceinstructionsel
\textbf{Teil I} &\\
\danceinstructionsel
1 -- 8 	& Ohne die Fassung zu lösen, umrundet die Dame einmal ihren Herrn. Beide heben dabei ihre Arme etwas an. Am Ende muss die Dame sich einmal um sich selbst drehen, um wieder in die Kiekbuschfassung zu gehen.\\
%\danceinstructionsel
\textbf{Teil I} &\\
%\danceinstructionsel
\textbf{Teil II} &\\
\danceinstructionsel
\textbf{Teil I} &\\
%\danceinstructionsel
1 -- 8 	& Die Tänzer lösen die rechten Hände und die Dame umrundet ihren Herrn an der linken Hand, um sich zu dem Herrn des nachfolgenden Paares zu begeben. Dort dreht sie sich in die Kiekbuschfassung ein.
\danceeasymarker
\danceinstructionsend


%%%%%%%%%%%%%%%%%%%%%%%%%%%%%%%%%%%%%%%%%%%%%%%%%%%%%%%%%%%%%%%%%%%%%%%%%%%%%%%%
% Empty Page in the End for Layouting Multipage A4 Pages
%%%%%%%%%%%%%%%%%%%%%%%%%%%%%%%%%%%%%%%%%%%%%%%%%%%%%%%%%%%%%%%%%%%%%%%%%%%%%%%%

\newpage
% remove the background
\ClearShipoutPictureBG{}
\null  % to have some content (otherwise the page would not be displayed)

\end{document}
