\documentclass[
	12pt,
	%draft, % show overfull hboxes
	]{scrartcl}

\usepackage[    % set up dimensions
	a6paper, 
	landscape,
	left=1.45cm,   
	right=1.45cm,  
	top=1.6cm,
	bottom=1.6cm, 
	%showframe,
	nomarginpar,  % this removes the right margin, centering the text in the middle
	noheadfoot,  % this removes the head and foot margin
	]{geometry}

%\pagestyle{empty}

\usepackage{graphicx}
\usepackage{tikz}
\usetikzlibrary{external}
\usepackage{eso-pic}

% Fonts and encodings
\usepackage[nf]{coelacanth}
\usepackage[T1]{fontenc}

\usepackage{longtable} % Um Tabellen zu machen, die mehrere Seiten lang sind und einen passenden Seitenumbruch haben.
\usepackage{multicol}
%\usepackage[singlespacing]{setspace} % Zeilenabstand
\usepackage{nicefrac} % für besser aussehende Brüche

\usepackage[object=vectorian]{pgfornament}

% pstricks vs pgf/tikz:
% Originally we used pstricks to draw the background ornaments.
% This had the disadvantage of having to use `latex -> dvips -> ps2pdf` with the correct options to get the background
% graphics to display in the right orientation. The necessary options also varied between MiKTeX and TeXLive :-(.
%
% Since i have more experience with tikz, i decided to use that, because it was easier to position the
% elements (they weren't placed quite correctly with the pstricks version.).

% pgfornament vs eps files
% I wanted to use pgfornament at first because it has everything necessary in one easy package.
% Unfortunately the graphics in pgfornament end up having quite thick strokes. This does not look good at the scales
% we're using the patterns.
% So i downloaded the vectorian files and extracted the patterns manually with Inkscape. This probably leads to a slowdown though.


\AddToShipoutPictureBG{%
\begin{tikzpicture}[remember picture, overlay]
	% pgfornament does give very simple and nice vectorian graphics for tikz.
	% The problem is that the lines are too thick and figure 41 is also slightly cropped.
	% pst-vectorian works very well.
	% we're now using the original files from vectorian exported as eps file
	
	% DEBUGGING grid
	%\draw[line width=0.75pt, color=teal, step=.5cm, xshift=0.08cm, yshift=0.43cm] (current page.south west) grid ++(14.85cm,10.5cm);
	% DEBUGGING corners
	%\node [circle, fill=red, minimum size=1pt, xshift=0.0cm, yshift=-0.0cm, scale=0.2] at (current page.north west) {};
	%\node [circle, fill=red, minimum size=1pt, xshift=0.5cm, yshift=-0.5cm, scale=0.2] at (current page.north west) {};
	
	% TODO: This slows down compilation a lot.
	%       Maybe find a faster way to do this.
	\node[anchor=north west, xshift=0.5cm, yshift=-0.5cm, inner sep=0] at (current page.north west){\includegraphics[scale=0.8, angle=0, origin=c]{img/corner_black}};
	\node[anchor=north east, xshift=-0.5cm, yshift=-0.5cm, inner sep=0] at (current page.north east){\includegraphics[scale=0.8, angle=0, origin=c]{img/corner2_black}};
	\node[anchor=south west, xshift=0.5cm, yshift=0.5cm, inner sep=0] at (current page.south west){\includegraphics[scale=0.8, angle=180, origin=c]{img/corner2_black}};
	\node[anchor=south east, xshift=-0.5cm, yshift=0.5cm, inner sep=0] at (current page.south east){\includegraphics[scale=0.8, angle=180, origin=c]{img/corner_black}};
	
	
	% Original pgfornament style:
%	\node[anchor=north west, xshift=0.5cm, yshift=-0.5cm,inner sep=0] at (current page.north west){\pgfornament[scale=0.5]{41}};
%	\node[anchor=north east, xshift=-0.5cm, yshift=-0.5cm, inner sep=0] at (current page.north east){\pgfornament[scale=0.5,symmetry=v]{41}};
%	\node[anchor=south west, xshift=0.5cm, yshift=0.5cm, inner sep=0] at (current page.south west){\pgfornament[scale=0.5,symmetry=h]{41}};
%	\node[anchor=south east, xshift=-0.5cm, yshift=0.5cm, inner sep=0] at (current page.south east){\pgfornament[scale=0.5,symmetry=c]{41}};
	
	% Background as an image, for example a parchment like texture
	%\tikz[remember picture, overlay] \node[opacity=0.5, inner sep=0pt] at (current page.center){\includegraphics[width=\paperwidth,height=\paperheight]{example-image}};
	
\end{tikzpicture}
}

\newcommand{\danceinfo}[1]{%
	{\raggedleft\footnotesize{#1}\par}
}
\newcommand{\dancename}[1]{
	% Dimension are manually chosen so they don't collide with the ornaments in the corners.	
	\begin{tikzpicture}[remember picture, overlay]
%		\node[] at (0, 0) (title) {Text};		
		\node[anchor=north, yshift=-0.5cm, inner sep=0, text width=10.5cm, align=center] at (current page.north){\Huge{#1}};
	\end{tikzpicture}
}
\newcommand{\origininfo}[3]{
	% Dimension are manually chosen so they don't collide with the ornaments in the corners.
	\begin{tikzpicture}[remember picture, overlay]
		% setting the font size in the node options, since it doesn't seem to 
		% take effect when used in the node text. 
		\node[anchor=south west, xshift=1.3cm, yshift=1.15cm, inner sep=0, align=left, font={\scriptsize}] at (current page.south west){Choreographie:\\{#1}};
		\node[anchor=south east, xshift=-1.2cm, yshift=1.15cm, inner sep=0, align=right, font={\scriptsize}] at (current page.south east){Musik:\\ {#2}\\ {#3}};
	\end{tikzpicture}
}
\newcommand{\danceeasymarker}{
	\begin{tikzpicture}[remember picture, overlay]
		\node[anchor=south, xshift=0cm, yshift=0.5cm, inner sep=0] at (current page.south){\pgfornament[width=4cm]{80}};
	\end{tikzpicture}
}
\newcommand{\dancemediummarker}{
	\begin{tikzpicture}[remember picture, overlay]
		\node[anchor=south, xshift=0cm, yshift=0.5cm, inner sep=0] at (current page.south){\pgfornament[width=5cm]{83}};
	\end{tikzpicture}
}
\newcommand{\dancedifficultmarker}{
\begin{tikzpicture}[remember picture, overlay]
	\node[anchor=south, xshift=0cm, yshift=0.4cm, inner sep=0] at (current page.south){\pgfornament[width=5cm]{84}};
\end{tikzpicture}
}
\newcommand{\danceinstructionsbegin}{\begin{longtable}{p{1cm}p{9.8cm}}}
\newcommand{\danceinstructionsend}{\end{longtable}}
\newcommand{\danceinstructionsel}{~ & ~ \\}

\setlength{\topskip}{12pt}
\setlength{\parskip}{0pt}

\pagestyle{empty}

\begin{document}

%%%%%%%%%%%%%%%%%%%%%%%%%%%%%%%%%%%%%%%%%%%%%%%%%%%%%%%%%%%%%%%%%%%%%%%%%%%%%%%%
%	Indian Queen
%%%%%%%%%%%%%%%%%%%%%%%%%%%%%%%%%%%%%%%%%%%%%%%%%%%%%%%%%%%%%%%%%%%%%%%%%%%%%%%%
\newpage

\dancename{Indian Queen}
\iffalse\subsection{Indian Queen}\fi % fool texstudio into displaying subsections
\danceinfo{Longway for as many as will}

\danceinstructionsbegin
1 -- 4  & First Corner Set zum Partner dann zum Kontra\\
5 -- 8  & First Corner Turn rechts\\
9 -- 16 & First Corner Ronde\\
1 -- 8  & Second Corner Set \& Turn zum Partner, dann Kontra\\
9 -- 16 & Second Corner Ronde\\
\danceinstructionsel
1 -- 8  & Mühle rechts herum (rechte Hände) auf 8 Klatschen\\
9 -- 16 & Mühle links herum zurück, auf 8 Klatschen\\
\danceinstructionsel
1 -- 8  & Dos-à-dos\\
1 -- 8  & $\nicefrac{3}{4}$ Kette\\
\danceeasymarker
\danceinstructionsend


%%%%%%%%%%%%%%%%%%%%%%%%%%%%%%%%%%%%%%%%%%%%%%%%%%%%%%%%%%%%%%%%%%%%%%%%%%%%%%%%
% Mulberry Garden
%%%%%%%%%%%%%%%%%%%%%%%%%%%%%%%%%%%%%%%%%%%%%%%%%%%%%%%%%%%%%%%%%%%%%%%%%%%%%%%%
\newpage

\dancename{Mulberry Garden}
\iffalse\subsection{Mulberry Garden}\fi % fool texstudio into displaying subsections
\danceinfo{Longway for as many as will}
\danceinstructionsbegin
1--8 	& Doppel vor und zurück\\
9--16 	& Wiederholung\\
\danceinstructionsel
1--8 	& Herren und Damenreihen Fall back and Meet\\
1--8 	& Partner Ronde\\
\danceinstructionsel
1--8 	& Dos-a-Dos mit dem Partner\\
1--8 	& Dos-a-Dos auf der Linie\\
\danceinstructionsel
1--4 	& halber Setkreis\\
5--8 	& halbe Ronde mit Partner\\
1--8 	& Fontäne\\
\dancemediummarker
\danceinstructionsend


%%%%%%%%%%%%%%%%%%%%%%%%%%%%%%%%%%%%%%%%%%%%%%%%%%%%%%%%%%%%%%%%%%%%%%%%%%%%%%%%
% Childgrove
%%%%%%%%%%%%%%%%%%%%%%%%%%%%%%%%%%%%%%%%%%%%%%%%%%%%%%%%%%%%%%%%%%%%%%%%%%%%%%%%
\newpage

\dancename{Childgrove \normalsize{(Arbon)}}
\iffalse\subsection{Childgrove}\fi % fool texstudio into displaying subsections
\danceinfo{Longway for as many as will}
\danceinstructionsbegin
1 -- 8 	& Siding rechtsschultrig\\
9 -- 16	& Herren Dos-à-dos\\
1 -- 8	& Siding linksschultrig\\
9 -- 16	& Damen Dos-à-dos\\
\danceinstructionsel
1 -- 4	& P1 turn single rechts\\
5 -- 8	& Auf den Seiten Platzwechsel rechtsschultrig\\
9 -- 16	& P1 Ronde (Diese 16 Schläge sollten für D1 fließend ineinander übergehen.)\\
1 -- 16	& Verwobene „Figure 8“\\
& \textit{(1er cross up, cast down, cross up, cast down - 2er cast down, cross up, cast down, cross up)}\\
\dancemediummarker
\danceinstructionsend


%%%%%%%%%%%%%%%%%%%%%%%%%%%%%%%%%%%%%%%%%%%%%%%%%%%%%%%%%%%%%%%%%%%%%%%%%%%%%%%%
% Hole in the Wall
%%%%%%%%%%%%%%%%%%%%%%%%%%%%%%%%%%%%%%%%%%%%%%%%%%%%%%%%%%%%%%%%%%%%%%%%%%%%%%%%
\newpage

\dancename{Hole in the Wall}
\iffalse\subsection{Hole in the Wall}\fi % fool texstudio into displaying subsections
\danceinfo{Longway for as many as will\\¾ Takt, Walzerschritte}
\danceinstructionsbegin
1--12 	& Paar 1 Cast down \& Lead up\\
1--12 	& Paar 2 Cast up \& Lead down\\
~ & ~ \\
1--6 	& FC Platzwechsel rechte Hand, kurz in der Mitte verharren\\
7--12	& SC Platzwechsel linke Hand, kurz in der Mitte verharren\\
~ & ~ \\
1--6 	& Halber Setkreis\\
7--12	& Paar 1 Cast down und zieht Paar 2 hoch \\
\danceeasymarker
\danceinstructionsend


%%%%%%%%%%%%%%%%%%%%%%%%%%%%%%%%%%%%%%%%%%%%%%%%%%%%%%%%%%%%%%%%%%%%%%%%%%%%%%%%
% Juice of Barley (Variante)
% Erläuterungen: abgewandelt,
% Notes:
% Used in the 1996 Television production of Emma.

% Original: 
% The Juice of Barley	Dancing Master 1688
% The 1 cu go back to back with their Partners, and the 2 cu do the same at the same time.
% The 1 cu take hands with his Partner and turn her round, the 2 cu doing the same at the same time.
% The two we stand still, whilst the 1 man goes round about the 2 wo into the 2 man’s place, and the 2 man goes round about the 1 wo into the 1 man’s place, 
% then all clap hands, then all four take hands and go quite round, the we doing the like. 
%%%%%%%%%%%%%%%%%%%%%%%%%%%%%%%%%%%%%%%%%%%%%%%%%%%%%%%%%%%%%%%%%%%%%%%%%%%%%%%%
\newpage

\dancename{Juice of Barley}
\iffalse\subsection{Juice of Barley (Variante)}\fi % fool texstudio into displaying subsections
\origininfo{Playford abgewandelt}{Playford}{}
\danceinfo{Longway for as many as will}
\danceinstructionsbegin
1 -- 8 &	Paare Dos-á-Dos \\
		& umrunden + vertreiben\\
1 -- 8 &	H1 durch Damen, um D2 auf Platz H2 \\
9 -- 16 &	H2 durch Damen, um D1 auf Platz H1 \\
1 -- 8 &	D1 durch Herren, um H1 auf Platz D2 \\
9 -- 16 &	D2 durch Herren, um H2 auf Platz D1 \\
1 -- 4 &	P1 geht gefasst nach oben (verabschieden)\\
5 -- 8 & 	und wieder zurück\\
\dancemediummarker
\danceinstructionsend


%%%%%%%%%%%%%%%%%%%%%%%%%%%%%%%%%%%%%%%%%%%%%%%%%%%%%%%%%%%%%%%%%%%%%%%%%%%%%%%%
% Jamaica
%%%%%%%%%%%%%%%%%%%%%%%%%%%%%%%%%%%%%%%%%%%%%%%%%%%%%%%%%%%%%%%%%%%%%%%%%%%%%%%%
\newpage

\dancename{Jamaica}
\iffalse\subsection{Jamaica}\fi % fool texstudio into displaying subsections
\danceinfo{Longway, Doppelter Fortschritt}
\danceinstructionsbegin
\textbf{I}& \\
1--4 	& P1 „Rechte Hand und linke Hand“(geben)\\
5--8 	& P1 „Auf die andere Seite“ (Platzwechsel)\\
9--16 	& H1+D2 / H2+D1 „Rechte Hand und linke Hand“ + Platzwechsel\\
%\danceinstructionsel
1--12 	& P1 Figure of Eight um P2 \\
13--16	& P1 halbe Ronde (ergibt sich aus Figure of Eight)\\
& \textit{Erster Fortschritt (Es gibt neue Sets)}\\
\textbf{II} & \\
1--8 	& FC Ronde\\
9--16	& SC Ronde\\
1--8 	& H1+H2, D1+D2  1½ Ronde\\
9--16 	& Dos-á-Dos
\dancemediummarker
\danceinstructionsend


%%%%%%%%%%%%%%%%%%%%%%%%%%%%%%%%%%%%%%%%%%%%%%%%%%%%%%%%%%%%%%%%%%%%%%%%%%%%%%%%
% An Dro
%%%%%%%%%%%%%%%%%%%%%%%%%%%%%%%%%%%%%%%%%%%%%%%%%%%%%%%%%%%%%%%%%%%%%%%%%%%%%%%%
\newpage

\iffalse\subsection{An Dro}\fi % fool texstudio into displaying subsections

\dancename{An Dro}
\danceinfo{Kette}

~\\
\textit{Kleine Finger haken bei den Nachbarn ein, Arme gesenkt und leicht gebeugt.}
\danceinstructionsbegin
1 -- 4 &	Double links, Hände malen eine 6 aufwärts\\
5 -- 8 &	Double rechts, Hände malen eine 9 abwärts\\
\danceeasymarker
\danceinstructionsend
\textit{Hinweis: Double links wird größer getanzt um vorwärts zu kommen}


%%%%%%%%%%%%%%%%%%%%%%%%%%%%%%%%%%%%%%%%%%%%%%%%%%%%%%%%%%%%%%%%%%%%%%%%%%%%%%%%
% The Crossroads
% Video: https://www.youtube.com/watch?v=t_mkdMxuZ6Y
%%%%%%%%%%%%%%%%%%%%%%%%%%%%%%%%%%%%%%%%%%%%%%%%%%%%%%%%%%%%%%%%%%%%%%%%%%%%%%%%
\newpage

\iffalse\subsection{The Crossroads}\fi % fool texstudio into displaying subsections
\dancename{The Crossroads}
\danceinfo{Tanz für 4 Paare,\\P1\&P2 improper}
\origininfo{Sascha Glimmann}{Soldier Boy}{von Custer LaRue}
\danceinstructionsbegin
1 -- 4 &  Set links zum Partner\\
5 -- 8 &  Halber Ronde\\
9 -- 16 & Paare Dos-á-Dos\\
1 -- 8 &  Setkreis je 2 Paare\\
9 -- 16 & Außenpaare gaten Innenpaare nach außen \nicefrac{3}{4} herum, sodass der Longway um 90° gedreht endet.\\
\dancemediummarker
\danceinstructionsend


%%%%%%%%%%%%%%%%%%%%%%%%%%%%%%%%%%%%%%%%%%%%%%%%%%%%%%%%%%%%%%%%%%%%%%%%%%%%%%%%
% Scales of Justice
%%%%%%%%%%%%%%%%%%%%%%%%%%%%%%%%%%%%%%%%%%%%%%%%%%%%%%%%%%%%%%%%%%%%%%%%%%%%%%%%
% Source for Instructions:
% Tanzbuch Oliver Herde
% Source for Tune:
\newpage

\dancename{Scales of Justice}
\iffalse\subsection{Scales of Justice}\fi % fool texstudio into displaying subsections
\danceinfo{Longway for as many as will}
\origininfo{Chris Sacket \& Brooke Friendly}{Scale of Justice}{Shira Kammen \& Roguery}
\danceinstructionsbegin
1 -- 12 & \nicefrac{3}{4} Kette auf Seitenlinie beginnend\\
\danceinstructionsel
13--16 	& linke Handtour \nicefrac{3}{4} herum\\
1 -- 4 	& Partner lösen Handfassung. Die Inneren tanzen eine rechte \nicefrac{3}{4} Handtour. Die Äußeren gehen \nicefrac{1}{4} auf der Kreisbahn weiter.\\
5 -- 8 	& Paare geben sich die linken Hände zu einer ganzen Handtour\\
9 -- 12	& Die Inneren tanzen wieder \nicefrac{3}{4}, die Äußeren gehen \nicefrac{1}{4} weiter.\\
13--16	& Linke Handtour \nicefrac{3}{4} zurück in die Gasse.\\
\danceinstructionsel
\danceinstructionsel
1 -- 12	& 3x Cecil Sharp Siding links, rechts dann linksschultrig\\
13--16	& Gelaufener Turn rechts\\
1 -- 8 	& Blende\\
9 -- 16	& Ganze Linkhand-Mühle
\dancemediummarker
\danceinstructionsend


%%%%%%%%%%%%%%%%%%%%%%%%%%%%%%%%%%%%%%%%%%%%%%%%%%%%%%%%%%%%%%%%%%%%%%%%%%%%%%%%
% Dargason
%%%%%%%%%%%%%%%%%%%%%%%%%%%%%%%%%%%%%%%%%%%%%%%%%%%%%%%%%%%%%%%%%%%%%%%%%%%%%%%%
\newpage

\dancename{Dargason} 
\origininfo{Playford}{Playford}{}
\iffalse \subsection{Dargason}\fi % fool texstudio into displaying subsections
\danceinfo{Tanz für 4 Paare\\Aufstellung D D D D H H H H\\Alle schauen in die Mitte}
\textit{Die Sidings werden meist als Cecil Sharp Siding ausgeführt\\(ähnlich einem halben Gypsy)}
\danceinstructionsbegin
\textbf{Teil} & \textbf{A1}:\\
1 -- 8 & Erstes Paar in der Mitte: Siding links\\
1 -- 8 & Set links \& Passieren: Beide passieren mit ganzem Turn rechts, wobei sie rechts voneinander gehen. Man endet mit selber Blickrichtung wie vor dem Turn.\\
1 -- 16	& Die mittleren zwei Paare: Siding links, Set links \& Passieren\\
1 -- 16	& Die mittleren drei Paare: Siding links, Set links \& Passieren\\
1 -- 16	& Alle Paare: Siding links - Set links \& Passieren. Jetzt ist das 1.mittlere Paar auf der Gegenseite hinten angekommen und bleibt stehen\\
1 -- 16 & Die mittleren drei Paare: Siding links - Set links, passieren, stehen\\
1 -- 16	& Die mittleren zwei Paare: Siding links - Set links, passieren, stehen\\
1 -- 16	& Das mittlere Paar: Siding links - Set links, passieren. Alle stehen wieder in Reihe auf Gegenseite, sofort:\\
\textbf{Teil} & \textbf{B1}:\\
1 -- 16	& Mittleres Paar: Arming rechts halb herum und zurück, Set links \& Passieren\\
1 -- 16	& die mittleren 2 Paare: Arming rechts halb herum und zurück, Set links \& Passieren\\
1 -- 16	& die mittleren 3 Paare: Arming rechts ganz herum - Set links \& Passieren\\
1 -- 16	& alle 4 Paare: Arming rechts ganz herum - Set links \& Passieren. Jetzt ist das 1.mittlere Paar auf der Gegenseite hinten angekommen und bleibt stehen\\
1 -- 64	& zurück bis zum eigenen Platz wie am Anfang für alle. Alle stehen wieder in Reihe\\
\textbf{Teil} & \textbf{C1}:\\
1 -- 32	& Mittlere Paar beginnt eine Hey und rechter Hand. Diese wird ganz durch getanzt.\\
\dancedifficultmarker
\danceinstructionsend
%\dancename{Dargason} 
%\origininfo{Playford}{Playford}{}
%\iffalse \subsection{Dargason}\fi % fool texstudio into displaying subsections
%\danceinfo{Longway for as many as will\\Aufstellung: . . . D D D H H H . . . \\Alle schauen in die Mitte}
%\textit{Die Sidings werden meist als Cecil Sharp Siding ausgeführt\\(ähnlich einem halben Gypsy)}
%\danceinstructionsbegin
%\textbf{Strophe} & \textbf{I}\\
%1 -- 8 & Erstes Paar in der Mitte: Siding links\\
%1 -- 8 & Set links \& Passieren: Beide passieren mit ganzem Turn rechts, wobei sie rechts voneinander gehen. Man endet mit selber Blickrichtung wie vor dem Turn.\\
%1 -- 16 & Die mittleren zwei Paare: Siding links, Set links \& Passieren\\
%1 -- 16	& Die mittleren drei Paare: Siding links, ... \\
%\danceinstructionsel
%1 -- 16 & \hspace*{1.0cm} $\vdots$\\
% & \textit{So Tanzen bis die innersten Tänzer außen angekommen sind. Dann wieder den selben Weg zurücktanzen bis alle wieder an ihrer Startposition angelangt sind.}\\
%1 -- 16 & \hspace*{1.0cm} $\vdots$\\
%1 -- 16 & Die mittleren zwei Paare: Siding links, Set links \& Passieren\\
%1 -- 16 & Das mittlere Paar: Siding links, Set links \& Passieren. Alle sind nun wieder auf ihren Startpositionen.\\
%\danceinstructionsel
%\textbf{Strophe} & \textbf{II}\\
%1 -- XX & Wie auch in Strophe I aber statt des Sidings eine Armtour mit eingehakten Armen\\
%\danceinstructionsel
%\textbf{Strophe} & \textbf{III}\\
%1 -- XX & Eine Hey: Die beiden inneren Partner Starten eine Hey zum Partner mit der rechten Hand. Während sie nach außen gehen Tanzen so immer mehr in der Hey bis alle wieder auf ihren eigenen Plätzen sind.\\
%
%\dancemediummarker
%\danceinstructionsend
%
%\textit{Je nach Länge der Musik können mehr oder weniger Tänzer mittanzen beziehungsweise weniger oder mehr Strophen getanzt werden}\\


%%%%%%%%%%%%%%%%%%%%%%%%%%%%%%%%%%%%%%%%%%%%%%%%%%%%%%%%%%%%%%%%%%%%%%%%%%%%%%%%
% Softly Good Tummas
%%%%%%%%%%%%%%%%%%%%%%%%%%%%%%%%%%%%%%%%%%%%%%%%%%%%%%%%%%%%%%%%%%%%%%%%%%%%%%%%
\newpage

\dancename{Softly Good Tummas}
\iffalse\subsection{Softly Good Tummas}\fi % fool texstudio into displaying subsections
\danceinfo{Longway for as many as will}
\danceinstructionsbegin
1 -- 8 	& Halber Single file circle links (UZS)\\
9 -- 12 & Meet in der Setmitte\\
13--16 	& Turn links zurück -- auf 16 klatschen\\
1 -- 16 & Wiederholung rechts\\
\danceinstructionsel
1 -- 4 	& Paar 1 Cast down, Paar 2 lead up\\
5 -- 8 	& Setting steps links, rechts\\
9 -- 16 & Halbe Kette\\
\danceinstructionsel
1 -- 4 	& Setting steps zurück (Anlauf nehmen)\\
5 -- 8 	& Platzwechsel mit dem Partner\\
9 -- 16 & Paar 1 rondiert nach unten, Past 2 cast up
\dancemediummarker
\danceinstructionsend


%%%%%%%%%%%%%%%%%%%%%%%%%%%%%%%%%%%%%%%%%%%%%%%%%%%%%%%%%%%%%%%%%%%%%%%%%%%%%%%%
% Candles in the Dark
%%%%%%%%%%%%%%%%%%%%%%%%%%%%%%%%%%%%%%%%%%%%%%%%%%%%%%%%%%%%%%%%%%%%%%%%%%%%%%%%
\newpage

\dancename{Candles in the Dark}
\origininfo{Loretta Holz (2006)}{Candles in the Dark}{von Jonathan Jensen}
\iffalse\subsection{Candles in the Dark}\fi % fool texstudio into displaying subsections
\danceinfo{Longway for as many as will\\¾ Takt, Walzerschritte}
\danceinstructionsbegin
1--12 	& H1 D1 geführte Half-Figure Eight durch das andere Paar\\
1--12 	& H1 D2 geführte Half-Figure Eight durch das andere Paar\\
1--12 	& H2 D1 geführte Half-Figure Eight durch das andere Paar\\
1--12 	& H2 D2 geführte Half-Figure Eight durch das andere Paar\\
\danceinstructionsel
1--12 	& Mirror Dos-à-Dos auf der Linie - Paar 2 innen \\
1--12 	& ganzer Setkreis\\
\danceinstructionsel
1--12 	& Mirror Dos-à-Dos auf der Linie – Paar 1 innen\\
1--12 	& 1 ½ Ronden mit dem Partner\\
\dancemediummarker
\danceinstructionsend


%%%%%%%%%%%%%%%%%%%%%%%%%%%%%%%%%%%%%%%%%%%%%%%%%%%%%%%%%%%%%%%%%%%%%%%%%%%%%%%%
% Empty Page in the End for Layouting Multipage A4 Pages
%%%%%%%%%%%%%%%%%%%%%%%%%%%%%%%%%%%%%%%%%%%%%%%%%%%%%%%%%%%%%%%%%%%%%%%%%%%%%%%%

\newpage
% remove the background
\ClearShipoutPictureBG{}
\null  % to have some content (otherwise the page would not be displayed)
  
\end{document}
