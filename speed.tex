\documentclass[
	12pt,
	%draft, % show overfull hboxes
	]{scrartcl}

\usepackage[    % set up dimensions
	a6paper, 
	landscape,
	left=1.45cm,   
	right=1.45cm,  
	top=1.6cm,
	bottom=1.6cm, 
	%showframe,
	nomarginpar,  % this removes the right margin, centering the text in the middle
	noheadfoot,  % this removes the head and foot margin
	]{geometry}

%\pagestyle{empty}

\usepackage{graphicx}
\usepackage{tikz}
\usetikzlibrary{external}
\usepackage{eso-pic}

% Fonts and encodings
\usepackage[nf]{coelacanth}
\usepackage[T1]{fontenc}

\usepackage{longtable} % Um Tabellen zu machen, die mehrere Seiten lang sind und einen passenden Seitenumbruch haben.
\usepackage{multicol}
%\usepackage[singlespacing]{setspace} % Zeilenabstand
\usepackage{nicefrac} % für besser aussehende Brüche

\usepackage[object=vectorian]{pgfornament}

% pstricks vs pgf/tikz:
% Originally we used pstricks to draw the background ornaments.
% This had the disadvantage of having to use `latex -> dvips -> ps2pdf` with the correct options to get the background
% graphics to display in the right orientation. The necessary options also varied between MiKTeX and TeXLive :-(.
%
% Since i have more experience with tikz, i decided to use that, because it was easier to position the
% elements (they weren't placed quite correctly with the pstricks version.).

% pgfornament vs eps files
% I wanted to use pgfornament at first because it has everything necessary in one easy package.
% Unfortunately the graphics in pgfornament end up having quite thick strokes. This does not look good at the scales
% we're using the patterns.
% So i downloaded the vectorian files and extracted the patterns manually with Inkscape. This probably leads to a slowdown though.


\AddToShipoutPictureBG{%
\begin{tikzpicture}[remember picture, overlay]
	% pgfornament does give very simple and nice vectorian graphics for tikz.
	% The problem is that the lines are too thick and figure 41 is also slightly cropped.
	% pst-vectorian works very well.
	% we're now using the original files from vectorian exported as eps file
	
	% DEBUGGING grid
	%\draw[line width=0.75pt, color=teal, step=.5cm, xshift=0.08cm, yshift=0.43cm] (current page.south west) grid ++(14.85cm,10.5cm);
	% DEBUGGING corners
	%\node [circle, fill=red, minimum size=1pt, xshift=0.0cm, yshift=-0.0cm, scale=0.2] at (current page.north west) {};
	%\node [circle, fill=red, minimum size=1pt, xshift=0.5cm, yshift=-0.5cm, scale=0.2] at (current page.north west) {};
	
	% TODO: This slows down compilation a lot.
	%       Maybe find a faster way to do this.
	\node[anchor=north west, xshift=0.5cm, yshift=-0.5cm, inner sep=0] at (current page.north west){\includegraphics[scale=0.8, angle=0, origin=c]{img/corner_black}};
	\node[anchor=north east, xshift=-0.5cm, yshift=-0.5cm, inner sep=0] at (current page.north east){\includegraphics[scale=0.8, angle=0, origin=c]{img/corner2_black}};
	\node[anchor=south west, xshift=0.5cm, yshift=0.5cm, inner sep=0] at (current page.south west){\includegraphics[scale=0.8, angle=180, origin=c]{img/corner2_black}};
	\node[anchor=south east, xshift=-0.5cm, yshift=0.5cm, inner sep=0] at (current page.south east){\includegraphics[scale=0.8, angle=180, origin=c]{img/corner_black}};
	
	
	% Original pgfornament style:
%	\node[anchor=north west, xshift=0.5cm, yshift=-0.5cm,inner sep=0] at (current page.north west){\pgfornament[scale=0.5]{41}};
%	\node[anchor=north east, xshift=-0.5cm, yshift=-0.5cm, inner sep=0] at (current page.north east){\pgfornament[scale=0.5,symmetry=v]{41}};
%	\node[anchor=south west, xshift=0.5cm, yshift=0.5cm, inner sep=0] at (current page.south west){\pgfornament[scale=0.5,symmetry=h]{41}};
%	\node[anchor=south east, xshift=-0.5cm, yshift=0.5cm, inner sep=0] at (current page.south east){\pgfornament[scale=0.5,symmetry=c]{41}};
	
	% Background as an image, for example a parchment like texture
	%\tikz[remember picture, overlay] \node[opacity=0.5, inner sep=0pt] at (current page.center){\includegraphics[width=\paperwidth,height=\paperheight]{example-image}};
	
\end{tikzpicture}
}

\newcommand{\danceinfo}[1]{%
	{\raggedleft\footnotesize{#1}\par}
}
\newcommand{\dancename}[1]{
	% Dimension are manually chosen so they don't collide with the ornaments in the corners.	
	\begin{tikzpicture}[remember picture, overlay]
%		\node[] at (0, 0) (title) {Text};		
		\node[anchor=north, yshift=-0.5cm, inner sep=0, text width=10.5cm, align=center] at (current page.north){\Huge{#1}};
	\end{tikzpicture}
}
\newcommand{\origininfo}[3]{
	% Dimension are manually chosen so they don't collide with the ornaments in the corners.
	\begin{tikzpicture}[remember picture, overlay]
		% setting the font size in the node options, since it doesn't seem to 
		% take effect when used in the node text. 
		\node[anchor=south west, xshift=1.3cm, yshift=1.15cm, inner sep=0, align=left, font={\scriptsize}] at (current page.south west){Choreographie:\\{#1}};
		\node[anchor=south east, xshift=-1.2cm, yshift=1.15cm, inner sep=0, align=right, font={\scriptsize}] at (current page.south east){Musik:\\ {#2}\\ {#3}};
	\end{tikzpicture}
}
\newcommand{\danceeasymarker}{
	\begin{tikzpicture}[remember picture, overlay]
		\node[anchor=south, xshift=0cm, yshift=0.5cm, inner sep=0] at (current page.south){\pgfornament[width=4cm]{80}};
	\end{tikzpicture}
}
\newcommand{\dancemediummarker}{
	\begin{tikzpicture}[remember picture, overlay]
		\node[anchor=south, xshift=0cm, yshift=0.5cm, inner sep=0] at (current page.south){\pgfornament[width=5cm]{83}};
	\end{tikzpicture}
}
\newcommand{\dancedifficultmarker}{
\begin{tikzpicture}[remember picture, overlay]
	\node[anchor=south, xshift=0cm, yshift=0.4cm, inner sep=0] at (current page.south){\pgfornament[width=5cm]{84}};
\end{tikzpicture}
}
\newcommand{\danceinstructionsbegin}{\begin{longtable}{p{1cm}p{9.8cm}}}
\newcommand{\danceinstructionsend}{\end{longtable}}
\newcommand{\danceinstructionsel}{~ & ~ \\}

\setlength{\topskip}{12pt}
\setlength{\parskip}{0pt}

\pagestyle{empty}

\begin{document}
	
%%%%%%%%%%%%%%%%%%%%%%%%%%%%%%%%%%%%%%%%%%%%%%%%%%%%%%%%%%%%%%%%%%%%%%%%%%%%%%%%
% A Celt's New Dance
%%%%%%%%%%%%%%%%%%%%%%%%%%%%%%%%%%%%%%%%%%%%%%%%%%%%%%%%%%%%%%%%%%%%%%%%%%%%%%%%
%\iffalse
%\newpage
%
%\iffalse\subsection{[A Celt's New Dance]}\fi % fool texstudio into displaying subsections
%
%\dancename{A Celt's New Dance}
%\danceinfo{Grand Square}
%\danceinstructionsbegin
%1 -- 8 & Durchgefasst zum Kreis Meet \& Fall back\\
%9 -- 16 & zum Partner wenden 2-Hände Kette, dann Referenz zum neuen Partner.\\
%1 -- 16 & Wiederholen (Man endet wieder beim eigenen Partner)\\
%\danceinstructionsel
%
%\danceinstructionsend
%%\dancelengthmarker{6}
%%\dancedifficultmarker
%\fi

%%%%%%%%%%%%%%%%%%%%%%%%%%%%%%%%%%%%%%%%%%%%%%%%%%%%%%%%%%%%%%%%%%%%%%%%%%%%%%%%
% A' Rovin
%%%%%%%%%%%%%%%%%%%%%%%%%%%%%%%%%%%%%%%%%%%%%%%%%%%%%%%%%%%%%%%%%%%%%%%%%%%%%%%%
\newpage

\iffalse\subsection{A' Rovin}\fi % fool texstudio into displaying subsections

\dancename{A' Rovin}

\danceinfo{Longway for as many as will}
\danceinstructionsbegin
1 -- 8 	& Paar 1 Lead Down \& Cast Up (lebhaft gesprungen)\\
9 -- 16	& Paar 2 Lead Up \& Cast Down (lebhaft gesprungen)\\
\danceinstructionsel
1 -- 8 	& Dos-á-dos\\
9 -- 16	& Set Rechts zurück und Turn rechts wieder rein\\
\danceinstructionsel
1 -- 8 	& \nicefrac{3}{4} Kette
\danceinstructionsend
\danceeasymarker


%%%%%%%%%%%%%%%%%%%%%%%%%%%%%%%%%%%%%%%%%%%%%%%%%%%%%%%%%%%%%%%%%%%%%%%%%%%%%%%%
% A Trip to Kilburn
%%%%%%%%%%%%%%%%%%%%%%%%%%%%%%%%%%%%%%%%%%%%%%%%%%%%%%%%%%%%%%%%%%%%%%%%%%%%%%%%
\newpage

\dancename{A Trip to Kilburn}
%\subsection{A Trip to Kilburn}
\iffalse\subsection{A Trip to Kilburn}\fi % fool texstudio into displaying subsections
\danceinfo{Longway for as many as will}
\danceinstructionsbegin
1 -- 4 	& FC Settingsteps rechts aufeinander zu\\
5 -- 8 	& FC Settingsteps zurück\\
9 -- 12 & FC Platzwechsel, Klatschen beim dritten Schritt (auf 11)\\
13-- 16 & FC Turn links\\
\danceinstructionsel
1 -- 16 & SC Settingsteps links, Platzwechel, Turn links\\
\danceinstructionsel
1 -- 8 	& Halber Setkreis\\
9 -- 16 & Dos-à-dos\\
\danceinstructionsel
1 -- 8 	& \nicefrac{3}{4} Kette\\
9 -- 16 & Ronde
\dancemediummarker
\danceinstructionsend



%%%%%%%%%%%%%%%%%%%%%%%%%%%%%%%%%%%%%%%%%%%%%%%%%%%%%%%%%%%%%%%%%%%%%%%%%%%%%%%%
% Allemande (Paartanz)
%%%%%%%%%%%%%%%%%%%%%%%%%%%%%%%%%%%%%%%%%%%%%%%%%%%%%%%%%%%%%%%%%%%%%%%%%%%%%%%%
\newpage

\dancename{Allemande}
%\subsection{Allemande}
\iffalse\subsection{Allemande (Paartanz)}\fi % fool texstudio into displaying subsections
\danceinfo{Paartanz}
\textit{Die Dame steht vor dem Herren in der Kiekbuschfassung}\\
\danceinstructionsbegin
1 -- 24 & 3x Chassé links und Chassé rechts, dann wenden\\
%\danceinstructionsel
1 -- 24 & 3x Chassé links und Chassé rechts zurück\\
& \textit{Linke Hände lösen, zueinander wenden}\\
\danceinstructionsel
1 -- 8 	& Das Paar tanzt seitwärts ein kleines Chassé links, dann ein Chassé rechts\\
1 -- 4 	& Schritt kick nach links und rechts\\
5 -- 8 	& Platzwechsel mit Handtour links\\
\danceinstructionsel
& \textit{Rechte Hände loslassen, linke Hände fassen}\\
1 -- 8 & Das Paar tanzt seitwärts ein kleines Chassé links, dann ein Chassé rechts\\
1 -- 4 	& Schritt kick nach rechts und links\\
5 -- 8 	& Die Dame dreht vor dem Herren und nimmt wieder Kiekbuschfassung an.
\dancemediummarker
\danceinstructionsend



%%%%%%%%%%%%%%%%%%%%%%%%%%%%%%%%%%%%%%%%%%%%%%%%%%%%%%%%%%%%%%%%%%%%%%%%%%%%%%%%
% An Improper Notion
%%%%%%%%%%%%%%%%%%%%%%%%%%%%%%%%%%%%%%%%%%%%%%%%%%%%%%%%%%%%%%%%%%%%%%%%%%%%%%%%
\newpage

\dancename{An Improper Notion}
\iffalse\subsection{An Improper Notion}\fi % fool texstudio into displaying subsections
\danceinfo{Longway for as many as will}
\danceinstructionsbegin
1 -- 8 	& Lead Up \& Down\\
1 -- 4 	& Paar 1 Cast Down, Paar 2 \textbf{Cross Up}\\
5 -- 8 	& Paar 1 Lead Up, Paar 2 Cast Down\\
\danceinstructionsel
1 -- 4 	& Set links\\
5 -- 8 	& Paar 1 Cast Down, Paar 2 Lead Up\\
1 -- 4 	& Set links\\
5 -- 8 	& Paar 2 Cast Down, Paar 1 Lead Up\\
9 -- 12 & Paar 1 Cast Down, Paar 2 Lead Up\\
\danceinstructionsel
1 -- 16 & Paar 1 Figure of Eight durch Paar 2\\
\danceinstructionsend
\textit{~~~~~Paar 2 ist abwechselnd proper und improper.}
\dancemediummarker


%%%%%%%%%%%%%%%%%%%%%%%%%%%%%%%%%%%%%%%%%%%%%%%%%%%%%%%%%%%%%%%%%%%%%%%%%%%%%%%%
% Empty Page in the End for Layouting Multipage A4 Pages
%%%%%%%%%%%%%%%%%%%%%%%%%%%%%%%%%%%%%%%%%%%%%%%%%%%%%%%%%%%%%%%%%%%%%%%%%%%%%%%%

\newpage
% remove the background
\ClearShipoutPictureBG{}
\null  % to have some content (otherwise the page would not be displayed)

\end{document}
